
\documentclass[11pt]{article} % use larger type; default would be 10pt

%%% ----------------------------------------------------------
%%% Addings
\input{../../latex_additions/__technical_additions}
%%% End of the addings
%%% ----------------------------------------------------------


\title{TP de Probabilités discrètes\\ -- version 1.1}
\date{\vspace{-5ex}}
\author{Dr M. GUEDJ}

\begin{document}
\maketitle

\renewcommand{\contentsname}{Table des Matières}
\maketitle

%%%%%%%%%%%%%%%%%%%%%%%%%%%%%% LICENCE CC %%%%%%%%%%%%%%%%%%%%%%%%%%%%
\newpage
\begin{center}
	\includegraphics[scale=0.5]{licence/licence_cc.png}
	
	\begin{small}
TP de Probabilités discrètes de Dr Michaël GUEDJ est mis à disposition selon les termes de la licence Creative Commons Attribution 4.0 International.
Fondé(e) sur une œuvre à 
		\url{https://github.com/michaelguedj/ens__proba}.
	\end{small}
\end{center}
\newpage
%%%%%%%%%%%%%%%%%%%%%%%%%%%%%%%%%%%%%%%%%%%%%%%%%%%%%%%%%%%%%%%%%%%%%%

\tableofcontents
%%%
%%%
%%%
%%%
%%%
%%%

\section{TP 1 -- Le paradoxe des anniversaires}



\subsection{Problème des anniversaires}
Dans une classe de 25 élèves, 
quelle est la probabilité $p$, que deux élèves (au moins), 
fêtent leurs anniversaires le même jour ? 


\subsection{Intuition}
En vous basant sur votre intuition, estimer la probabilité en question.


\subsection{Considérations préalables}
On suppose :
\begin{itemize}
	\item  Que les années comportent $365$ jours par an 
	(années bissextiles non considérées) ; 
	\item La distribution des naissances est uniforme ; 
	i.e. quelque soit le jour $i\in \{1, ..., 365\}$, 
	quelque soit l'élève $x$, 
	la probabilité que $x$ fête son anniversaire le jour $i$
	est de $\frac{1}{165}$.
\end{itemize}


\subsection{Résolution du problème des anniversaires}
On considère une classe de $n$ élèves.
%%%
\begin{enumerate}
	\item Soit $n=2$. Donner la probabilité $\bar{p}$ que les deux élèves de la
	classe ne fêtent pas leurs anniversaires le même jour.
	
	\item Idem pour $n=3$.
	
	\item Idem pour $n=k$.
	
	\item 
	Résoudre le problème des anniversaires
	(utiliser la relation : $p=1-\bar{p}$).
	
	
	\item Considérer les cas : $n=30$ ; $n=50$ et $n=80$.
\end{enumerate}

\subsection{Côté individu}
Soit un individu : \textit{Toto}, dans une classe de $n=25$ élèves.
Quelle est la probabilité qu'un (autre) élève soit né le même jour que Toto ?

\begin{enumerate}
	\item Si $n=2$, quelle est la probabilité $\bar{p}$ pour que l'autre élève
	soit né un autre jour que Toto ?
	
	\item Idem pour $n=3$.
	
	\item Idem pour $n=k$.
	
	\item Trouver la probabilité $p$ 
	qu'un (autre) élève soit né le même jour que Toto 
	(utiliser la relation $p = 1-\bar{p}$).
	
	\item Que donne $p$ pour $n=30$ ? $n=50$ ? $n=80$ ?
	
\end{enumerate}

\subsection{Simulation}
On modélise une classe de $n$ élèves par une
liste de $n$ éléments compris entre $1$ et $365$.
(On suppose que la liste est indexée de $0$ à $n-1$).
\begin{enumerate}[(A)]
	\item 
	L'algorithme, ci-après, 
	permet de
	tester si deux personnes, dans la classe, sont nés le même jour.
\begin{Verbatim}
test_anniversaire(classe : liste de taille n)
begin
  for i = 0, ..., n-1 do
    for j = i+1, ..., n-1 do 
      if classe[i] = classe[j] then 
        return True
      end if
  end for; end for
  return False
end
\end{Verbatim}


\begin{fLemma}
Au $i$-ième tour de la première boucle,
en supposant que l'algorithme ne s'est pas arrêté,
$$
\forall x\in\{0, ..., i-1 \}, 
\forall j\in\{x+1, ..., n-1\}, 
classe[i] \not= classe[j]
$$
\end{fLemma}
%
\begin{proof}
Par l'absurde, 
$$
\exists x\in\{0, ..., i-1 \}, 
\exists j\in\{x+1, ..., n-1\}, 
classe[x] = classe[j]
$$
Donc au tour de boucle $x$, 
l'algorithme s'est arrêté, 
ce qui contredit notre hypothèse.
\end{proof}

\begin{fTheorem}
Au $i$-ième tour de la première boucle,
en supposant que l'algorithme ne s'est pas arrêté,
$$
	\forall j\in \{0, ..., i-1\},
	classe[i] \not= classe[j]
$$
\end{fTheorem}
	%
\begin{proof}[Preuve]
Par le lemme précédant, 
$$
\forall x\in\{0, ..., i-1 \}, 
\forall j\in\{x+1, ..., n-1\}, 
classe[i] \not= classe[j]
$$
On pose :
$$
E(x):= \{x+1, ..., n-1\}
$$
On a ainsi : 
$$
\forall x\in\{0, ..., i-1 \}, 
\forall j\in E(x), 
classe[i] \not= classe[j]
$$
On remarque que :
$$E(0) \supset E(1) \supset ... \supset E(i-1)$$
(En effet : 
$$E(0) = \{1, ..., n-1\}$$
$$E(1) = \{2, ..., n-1\}$$
$$ ... $$
$$E(i-1) = \{i, ..., n-1\}$$
)

De plus, 
$$
i\in E(i-1)
$$
Donc, 
$$
\forall x\in\{0, ..., i-1\}, i\in E(x)
$$
On rappelle que :
$$
\forall x\in\{0, ..., i-1 \}, 
\forall j\in E(x), 
classe[i] \not= classe[j]
$$
D'où, 
$$
\forall x\in\{0, ..., i-1 \}, 
classe[x] \not= classe[i]
$$
\end{proof}
	
	\item 
L'algorithme, ci-après, 
permet de
tester si une personne, dans la classe, est né le même jour que Toto.
\begin{Verbatim}
test_anniversaire_toto(classe : liste de taille n)
begin
  // Toto est supposé en position 0 (sans perte de généralité)
  for i = 1, ..., n-1 do
    if classe[0] = classe[i] then 
      return True
    end if
  end for
  return False
end
\end{Verbatim}

\item 
Via implémentation, évaluer ``pratiquement''
la justesse des résultats des sections précédentes.
\end{enumerate}
\newpage
\section{TP 1 -- Le paradoxe des anniversaires -- Correction}



\subsection{Problème des anniversaires}
Dans une classe de 25 élèves, 
quelle est la probabilité $p$, que deux élèves (au moins), 
fêtent leurs anniversaires le même jour ? 


\subsection{Intuition}
En vous basant sur votre intuition, estimer la probabilité en question.


\subsection{Considérations préalables}
On suppose :
\begin{itemize}
	\item  Que les années comportent $365$ jours par an 
	(années bissextiles non considérées) ; 
	\item La distribution des naissances est uniforme ; 
	i.e. quelque soit le jour $i\in \{1, ..., 365\}$, 
	quelque soit l'élève $x$, 
	la probabilité que $x$ fête son anniversaire le jour $i$
	est de $\frac{1}{165}$.
\end{itemize}


\subsection{Résolution du problème des anniversaires}
On considère une classe de $n$ élèves.
%%%
\begin{enumerate}
	\item Soit $n=2$. Donner la probabilité $\bar{p}$ que les deux élèves de la
	classe ne fêtent pas leurs anniversaires le même jour.
	\textcolor{red}{
		$$\bar{p} = \frac{365\times 364}{365^2}$$
		\begin{enumerate}
			\item Nombre de possibilités satisfaisant la contrainte : $365\times 364$.
			\item Nombre total de possibilités : $365^2$. 
		\end{enumerate}
	}
	
	\item Idem pour $n=3$.
	\textcolor{red}{
		$$\bar{p} = 
		\frac{365\times 364 \times 363}{365^3}
		$$
		$$
		\big(
		\bar{p}=
		\frac{365\times (365-2+1) \times (365-3+1)}{365^3}
		\big)
		$$
		\begin{enumerate}
			\item Nombre de possibilités satisfaisant la contrainte : 
			$365\times 364\times 363$.
			\item Nombre total de possibilités : $365^3$. 
		\end{enumerate}
	}
	
	\item Idem pour $n=k$.
	\textcolor{red}{
		$$\bar{p} = \frac{365\times 364\times ... \times (365-k+1)}{365^k}$$
		$$\bar{p} =
		\prod_{i=1}^{k} \frac{(365-i+1)}{365^k}
		$$
		\begin{enumerate}
			\item Nombre de possibilités satisfaisant la contrainte : 
			$$365\times 364\times ... \times (365-k+1)$$
			\item Nombre total de possibilités : $365^k$. 
		\end{enumerate}
	}
	
	\item 
	Résoudre le problème des anniversaires
	(utiliser la relation : $p=1-\bar{p}$).
	
	\textcolor{red}{
		On a, pour $n=k$: 
		$$
		p = 1 - \prod_{i=1}^{k} \frac{(365-i+1)}{365^k}
		$$
		Soit, pour $n=25$ :
		$$p =
		1 - \prod_{i=1}^{25} \frac{(365-i+1)}{365^k}
		$$
		Soit : 
		$$ 0.56 < p < 0.57$$
	}
	
	\item Considérer les cas : $n=30$ ; $n=50$ et $n=80$.
	\textcolor{red}{
		\\
		Pour $n=30$,
		$$ 0.70 < p < 0.71$$
		Pour $n=50$,
		$$ 0.970 < p < 0.971$$
		Pour $n=80$,
		$$ p \approx 0.99991$$
	}
\end{enumerate}

\subsection{Côté individu}
Soit un individu : \textit{Toto}, dans une classe de $n=25$ élèves.
Quelle est la probabilité qu'un (autre) élève soit né le même jour que Toto ?

\begin{enumerate}
	\item Si $n=2$, quelle est la probabilité $\bar{p}$ pour que l'autre élève
	soit né un autre jour que Toto ?
	\textcolor{red}{
		$$\bar{p} = \frac{364}{365}$$
	}
	
	\item Idem pour $n=3$.
	\textcolor{red}{
		$$\bar{p} = \Big(\frac{364}{365} \Big)^2$$
	}
	
	\item Idem pour $n=k$.
	\textcolor{red}{
		$$\bar{p} = \Big(\frac{364}{365} \Big)^{k-1}$$
	}
	
	\item Trouver la probabilité $p$ 
	qu'un (autre) élève soit né le même jour que Toto 
	(utiliser la relation $p = 1-\bar{p}$).
	\textcolor{red}{
	\\
	Pour $n = k$, 
		$$
		p = 1 - \bar{p}
		= 1 - \Big(\frac{364}{365} \Big)^{k-1}
		$$
		Pour $n=25$
		$$
		p \approx 0.06
		$$
	}
	
	\item Que donne $p$ pour $n=30$ ? $n=50$ ? $n=80$ ?
	\textcolor{red}{
		\\
		Pour $n=30$,
		$$ p \approx 0.08$$
		Pour $n=50$,
		$$ p \approx 0.13$$
		Pour $n=80$,
		$$ p \approx 0.2$$
	}
	
\end{enumerate}

\subsection{Simulation}
On modélise une classe de $n$ élèves par une
liste de $n$ éléments compris entre $1$ et $365$.
(On suppose que la liste est indexée de $0$ à $n-1$).
\begin{enumerate}[(A)]
	\item 
	L'algorithme, ci-après, 
	permet de
	tester si deux personnes, dans la classe, sont nés le même jour.
\begin{Verbatim}
test_anniversaire(classe : liste de taille n)
begin
  for i = 0, ..., n-1 do
    for j = i+1, ..., n-1 do 
      if classe[i] = classe[j] then 
        return True
      end if
  end for; end for
  return False
end
\end{Verbatim}


\begin{fLemma}
Au $i$-ième tour de la première boucle,
en supposant que l'algorithme ne s'est pas arrêté,
$$
\forall x\in\{0, ..., i-1 \}, 
\forall j\in\{x+1, ..., n-1\}, 
classe[i] \not= classe[j]
$$
\end{fLemma}
%
\begin{proof}
Par l'absurde, 
$$
\exists x\in\{0, ..., i-1 \}, 
\exists j\in\{x+1, ..., n-1\}, 
classe[x] = classe[j]
$$
Donc au tour de boucle $x$, 
l'algorithme s'est arrêté, 
ce qui contredit notre hypothèse.
\end{proof}

\begin{fTheorem}
Au $i$-ième tour de la première boucle,
en supposant que l'algorithme ne s'est pas arrêté,
$$
	\forall j\in \{0, ..., i-1\},
	classe[i] \not= classe[j]
$$
\end{fTheorem}
	%
\begin{proof}[Preuve]
Par le lemme précédant, 
$$
\forall x\in\{0, ..., i-1 \}, 
\forall j\in\{x+1, ..., n-1\}, 
classe[i] \not= classe[j]
$$
On pose :
$$
E(x):= \{x+1, ..., n-1\}
$$
On a ainsi : 
$$
\forall x\in\{0, ..., i-1 \}, 
\forall j\in E(x), 
classe[i] \not= classe[j]
$$
On remarque que :
$$E(0) \supset E(1) \supset ... \supset E(i-1)$$
(En effet : 
$$E(0) = \{1, ..., n-1\}$$
$$E(1) = \{2, ..., n-1\}$$
$$ ... $$
$$E(i-1) = \{i, ..., n-1\}$$
)

De plus, 
$$
i\in E(i-1)
$$
Donc, 
$$
\forall x\in\{0, ..., i-1\}, i\in E(x)
$$
On rappelle que :
$$
\forall x\in\{0, ..., i-1 \}, 
\forall j\in E(x), 
classe[i] \not= classe[j]
$$
D'où, 
$$
\forall x\in\{0, ..., i-1 \}, 
classe[x] \not= classe[i]
$$
\end{proof}
	
	\item 
L'algorithme, ci-après, 
permet de
tester si une personne, dans la classe, est né le même jour que Toto.
\begin{Verbatim}
test_anniversaire_toto(classe : liste de taille n)
begin
  // Toto est supposé en position 0 (sans perte de généralité)
  for i = 1, ..., n-1 do
    if classe[0] = classe[i] then 
      return True
    end if
  end for
  return False
end
\end{Verbatim}

\item 
Via implémentation, évaluer ``pratiquement''
la justesse des résultats des sections précédentes.
\end{enumerate}

\newpage
\section{TP 2 -- Paradoxe des deux enfants}


%%%%%%%%%%%%%%%%%%%%%%%%%%%%%%%%%%%%%%%%%%%%%%%%%%%%%%%%
%%%%%%%%%%%%%%%%%%%%%%%%%%%%%%%%%%%%%%%%%%%%%%%%%%%%%%%%
%%%%%%%%%%%%%%%%%%%%%%%%%%%%%%%%%%%%%%%%%%%%%%%%%%%%%%%%
\subsection{Enoncé du problème}
Toto a exactement deux enfants.
Étant donné que l'un de ces deux enfants est un garçon,
et que la probabilité d'avoir un garçon est de $\frac{1}{2}$, 
quelle est la probabilité que l'autre enfant soit aussi un garçon ?

%%%%%%%%%%%%%%%%%%%%%%%%%%%%%%%%%%%%%%%%%%%%%%%%%%%%%%%%
%%%%%%%%%%%%%%%%%%%%%%%%%%%%%%%%%%%%%%%%%%%%%%%%%%%%%%%%
%%%%%%%%%%%%%%%%%%%%%%%%%%%%%%%%%%%%%%%%%%%%%%%%%%%%%%%%
\subsection{Résolution}
$$
\Pr(\text{2 garçons } | \text{ 1 garçon au moins}) =
 \frac {\Pr(\text{2 garçons } \wedge 
 			\text{ 1 garçon au moins})}
 		{\Pr(\text{1 garçon au moins})}
$$
$$
\Pr(\text{2 garçons } | \text{ 1 garçon au moins}) =
  \frac
 {\Pr(\text{2 garçons})}
 {\Pr(\text{1 garçon au moins})}
$$

$$
\Pr(\text{2 garçons } | \text{ 1 garçon au moins}) =
 \frac{1/4}{3/4} 
$$

$$
\Pr(\text{2 garçons } | \text{   1 garçon au moins}) =
 \frac{1}{3} 
$$


%%%%%%%%%%%%%%%%%%%%%%%%%%%%%%%%%%%%%%%%%%%%%%%%%%%%%%%%
%%%%%%%%%%%%%%%%%%%%%%%%%%%%%%%%%%%%%%%%%%%%%%%%%%%%%%%%
%%%%%%%%%%%%%%%%%%%%%%%%%%%%%%%%%%%%%%%%%%%%%%%%%%%%%%%%
\subsection{Exercice}
L'objectif est d'effectuer des simulations informatiques,
assurant (ou non), la fiabilité du résultat théorique.



\newpage
\section{TP3 -- Espérance}




\subsection{Espérance et lancers de pièces}
%%%%%%%%%%%%%%%%%%%%%%%%%%%%%%%%%%%%%%%%%%%%%%%%%%%%%%%%%%%%%%%%%%%%%%
%%%%%%%%%%%%%%%%%%%%%%%%%%%%%%%%%%%%%%%%%%%%%%%%%%%%%%%%%%%%%%%%%%%%%%
%%%%%%%%%%%%%%%%%%%%%%%%%%%%%%%%%%%%%%%%%%%%%%%%%%%%%%%%%%%%%%%%%%%%%%%%%%%%%%%%%%%%%%%%%%%%%%%%%%%%%%%%%%%%%%%%%%%%%%%%%%%%%%%%%%%%%%%%%%%%
\subsubsection*{Enoncé du problème}

On demande à 100 personnes 
de lancer (chacun) deux pièces de monnaie.
On comptabilise ensuite le nombre de piles.

Quel et le nombre de piles (``environ'') obtenu ?


De même mais avec 1 000 personnes ; puis 10 000 personnes.


%%%%%%%%%%%%%%%%%%%%%%%%%%%%%%%%%%%%%%%%%%%%%%%%%%%%%%%%
%%%%%%%%%%%%%%%%%%%%%%%%%%%%%%%%%%%%%%%%%%%%%%%%%%%%%%%%
%%%%%%%%%%%%%%%%%%%%%%%%%%%%%%%%%%%%%%%%%%%%%%%%%%%%%%%%
\subsubsection*{Explication théorique}

L'univers des événements pour un lancer est : 
$\Omega = \{PP, PF, FP, FF\}$
avec, en supposant 
$\Pr(P)=\Pr(F)=\frac{1}{2}$ :
$$
\Pr(PP) = \Pr(PF) = 
\Pr(FP) = \Pr(FF) = 
\frac{1}{4}
$$

La variable aléatoire $X$ quantifie le nombre de piles pour un lancer.
$$
X(\Omega) = \{0, 1, 2\}
$$

La loi de probabilité de $X$ est : 
$$
\mathcal{L}_X = \{
~~ 0 \rightarrow \frac{1}{4};
~~1 \rightarrow \frac{1}{2};
~~2 \rightarrow \frac{1}{4}
~~\}
$$

L'espérance de $X$ est ainsi :
$$\uE(X) = 0.\Pr(X=0) + 1.\Pr(X=1) + 2.\Pr(X=2)$$ 
$$\uE(X) = 0 + \frac{1}{2} + 2.\frac{1}{4}$$
$$\uE(X) = 1$$


Autrement dit, pour chaque lancer de 2 pièces, 
on peut ``espérer'' avoir 1 pile.

En pratique, cela signifie que si on effectue $n$ lancés,
alors :
si $n$ est ``grand'', le nombre de piles sera ``proche'' de $n$.


%%%%%%%%%%%%%%%%%%%%%%%%%%%%%%%%%%%%%%%%%%%%%%%%%%%%%%%%
%%%%%%%%%%%%%%%%%%%%%%%%%%%%%%%%%%%%%%%%%%%%%%%%%%%%%%%%
%%%%%%%%%%%%%%%%%%%%%%%%%%%%%%%%%%%%%%%%%%%%%%%%%%%%%%%%
\subsubsection*{Exercice}
L'objectif est d'effectuer des simulations informatiques,
assurant (ou non), la fiabilité du résultat théorique.




%%%%%%%%%%%%%%%%%%%%%%%%%%%%%%%%%%%%%%%%%%%%%%%%%%%%%%%%
%%%%%%%%%%%%%%%%%%%%%%%%%%%%%%%%%%%%%%%%%%%%%%%%%%%%%%%%
%%%%%%%%%%%%%%%%%%%%%%%%%%%%%%%%%%%%%%%%%%%%%%%%%%%%%%%%

%%%%%%%%%%%%%%%%%%%%%%%%%%%%%%%%%%%%%%%%%%%%%%%%%%%%%%%%
%%%%%%%%%%%%%%%%%%%%%%%%%%%%%%%%%%%%%%%%%%%%%%%%%%%%%%%%
%%%%%%%%%%%%%%%%%%%%%%%%%%%%%%%%%%%%%%%%%%%%%%%%%%%%%%%%

%%%%%%%%%%%%%%%%%%%%%%%%%%%%%%%%%%%%%%%%%%%%%%%%%%%%%%%%
%%%%%%%%%%%%%%%%%%%%%%%%%%%%%%%%%%%%%%%%%%%%%%%%%%%%%%%%
%%%%%%%%%%%%%%%%%%%%%%%%%%%%%%%%%%%%%%%%%%%%%%%%%%%%%%%%

%%%%%%%%%%%%%%%%%%%%%%%%%%%%%%%%%%%%%%%%%%%%%%%%%%%%%%%%
%%%%%%%%%%%%%%%%%%%%%%%%%%%%%%%%%%%%%%%%%%%%%%%%%%%%%%%%
%%%%%%%%%%%%%%%%%%%%%%%%%%%%%%%%%%%%%%%%%%%%%%%%%%%%%%%%



\subsection{Espérance et boîte magique}
%%%%%%%%%%%%%%%%%%%%%%%%%%%%%%%%%%%%%%%%%%%%%%%%%%%%%%%%%%%%%%%%%%%%%%
%%%%%%%%%%%%%%%%%%%%%%%%%%%%%%%%%%%%%%%%%%%%%%%%%%%%%%%%%%%%%%%%%%%%%%
%%%%%%%%%%%%%%%%%%%%%%%%%%%%%%%%%%%%%%%%%%%%%%%%%%%%%%%%%%%%%%%%%%%%%%%%%%%%%%%%%%%%%%%%%%%%%%%%%%%%%%%%%%%%%%%%%%%%%%%%%%%%%%%%%%%%%%%%%%%%
\subsubsection*{Enoncé du problème}
Soit une boîte magique 
agissant comme suit : 
\begin{itemize}
\item[--] La boîte "génère" : $3$, avec probabilité : $\frac{1}{3}$ ;
\item[--] La boîte "génère" : $-1$, avec probabilité : $\frac{2}{3}$.
\end{itemize}
Quelle est la somme des valeurs  
générées par la boîte, pour $n$ générations de valeur ?

%%%%%%%%%%%%%%%%%%%%%%%%%%%%%%%%%%%%%%%%%%%%%%%%%%%%%%%%
\subsubsection*{Explication théorique}

L'univers des événements pour une génération est : 
$$\Omega = \{-1, 3\}$$

Ici la variable aléatoire $X$ prend ses valeurs dans $X(\Omega) = \Omega$.

La loi de probabilité de $X$ est : 
$$
\mathcal{L}_X = \{
~~ -1 \rightarrow \frac{2}{3} ;
~~ 3 \rightarrow \frac{1}{3}
~~\}
$$

L'espérance de $X$ est ainsi :
$$\uE(X) = -1.\Pr(X=-1) + 3.\Pr(X=3) $$ 
$$\uE(X) = -1.\frac{2}{3} + 3.\frac{1}{3}$$
$$\uE(X) = \frac{1}{3}$$


Autrement dit, pour chaque valeur générée, 
on peut ``espérer'' avoir $\frac{1}{3}$.

En pratique, cela signifie que si on effectue $n$ 
générations de valeur,
alors :
si $n$ est "grand", 
la somme des valeurs générees sera ``proche'' de : 
$$n.\uE(X) = \frac{n}{3}$$

% \subsubsection{Prévision vs Expérience}

% Notons la prévsion : 
% $$p:=n.\uE(X)=\frac{n}{3}$$

% Après experience, on obtient une valeur de $r$.
% ($r$ pour ``réalisé'').

% Alors la variation $\Delta$ est définie par:
% $$
% \Delta := |p - r|
% $$

% On considère la moyenne empirique (i.e. l'espérance estimée) : 
% $$
% E_r := \frac{r}{n} (=\bar{x} )
% $$

% On pose $\Delta_E$ la variation des espérances : 
% $$
% \Delta_E := |E(X) - E_r|
% $$

% \blue{
% \begin{fTheorem}
% $$
% \Delta_E = \frac{\Delta}{n} 
% $$
% \end{fTheorem}
% }
% \begin{proof}
% $$\Delta_E = |E(X) - E_r|
% $$

% $$\Delta_E =  \frac{n.|E(X) - E_r|}{n}
% $$

% $$\Delta_E = \frac{|n.E(X) - n.E_r|}{n}
% $$

% Or :
% $$p = n.\uE(X)$$

% Et : 
% $$ E_r = \frac{r}{n} \iff  r = n.E_r $$

% $$\Delta_E = \frac{|p - r|}{n} = \frac{\Delta}{n}
% $$
% \end{proof}
%%%%%%%%%%%%%%%%%%%%%%%%%%%%%%%%%%%%%%%%%%%%%%%%%%%%%%%%
\subsubsection*{Exercice}
L'objectif est d'effectuer des simulations informatiques,
assurant (ou non), la fiabilité du résultat théorique.


\newpage
\section{TP4 -- Le paradoxe de Monty Hall}

$~$
\begin{itemize}
\item Soient trois portes :
 l'une cache une voiture, chacune des deux autres cachant une chèvre.
\item Le présentateur sait où se cache la voiture.
\item Le joueur, qui souhaite trouver la voiture, 
	choisit une des portes (sans que celle-ci ne soit ouverte).
\item Le présentateur ouvre une autre porte (que celle choisie par le joueur), qui révèle alors une chèvre.
\item Le présentateur demande au candidat si celui-ci souhaite modifier son choix, 
	avant que soit effectué l'ouverture des portes.
\end{itemize}



\subsubsection*{Exercice}
Que doit faire le joueur ?


\end{document}