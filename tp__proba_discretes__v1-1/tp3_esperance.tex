



\subsection{Espérance et lancers de pièces}
%%%%%%%%%%%%%%%%%%%%%%%%%%%%%%%%%%%%%%%%%%%%%%%%%%%%%%%%%%%%%%%%%%%%%%
%%%%%%%%%%%%%%%%%%%%%%%%%%%%%%%%%%%%%%%%%%%%%%%%%%%%%%%%%%%%%%%%%%%%%%
%%%%%%%%%%%%%%%%%%%%%%%%%%%%%%%%%%%%%%%%%%%%%%%%%%%%%%%%%%%%%%%%%%%%%%%%%%%%%%%%%%%%%%%%%%%%%%%%%%%%%%%%%%%%%%%%%%%%%%%%%%%%%%%%%%%%%%%%%%%%
\subsubsection*{Enoncé du problème}

On demande à 100 personnes 
de lancer (chacun) deux pièces de monnaie.
On comptabilise ensuite le nombre de piles.

Quel et le nombre de piles (``environ'') obtenu ?


De même mais avec 1 000 personnes ; puis 10 000 personnes.


%%%%%%%%%%%%%%%%%%%%%%%%%%%%%%%%%%%%%%%%%%%%%%%%%%%%%%%%
%%%%%%%%%%%%%%%%%%%%%%%%%%%%%%%%%%%%%%%%%%%%%%%%%%%%%%%%
%%%%%%%%%%%%%%%%%%%%%%%%%%%%%%%%%%%%%%%%%%%%%%%%%%%%%%%%
\subsubsection*{Explication théorique}

L'univers des événements pour un lancer est : 
$\Omega = \{PP, PF, FP, FF\}$
avec, en supposant 
$\Pr(P)=\Pr(F)=\frac{1}{2}$ :
$$
\Pr(PP) = \Pr(PF) = 
\Pr(FP) = \Pr(FF) = 
\frac{1}{4}
$$

La variable aléatoire $X$ quantifie le nombre de piles pour un lancer.
$$
X(\Omega) = \{0, 1, 2\}
$$

La loi de probabilité de $X$ est : 
$$
\mathcal{L}_X = \{
~~ 0 \rightarrow \frac{1}{4};
~~1 \rightarrow \frac{1}{2};
~~2 \rightarrow \frac{1}{4}
~~\}
$$

L'espérance de $X$ est ainsi :
$$\uE(X) = 0.\Pr(X=0) + 1.\Pr(X=1) + 2.\Pr(X=2)$$ 
$$\uE(X) = 0 + \frac{1}{2} + 2.\frac{1}{4}$$
$$\uE(X) = 1$$


Autrement dit, pour chaque lancer de 2 pièces, 
on peut ``espérer'' avoir 1 pile.

En pratique, cela signifie que si on effectue $n$ lancés,
alors :
si $n$ est ``grand'', le nombre de piles sera ``proche'' de $n$.


%%%%%%%%%%%%%%%%%%%%%%%%%%%%%%%%%%%%%%%%%%%%%%%%%%%%%%%%
%%%%%%%%%%%%%%%%%%%%%%%%%%%%%%%%%%%%%%%%%%%%%%%%%%%%%%%%
%%%%%%%%%%%%%%%%%%%%%%%%%%%%%%%%%%%%%%%%%%%%%%%%%%%%%%%%
\subsubsection*{Exercice}
L'objectif est d'effectuer des simulations informatiques,
assurant (ou non), la fiabilité du résultat théorique.




%%%%%%%%%%%%%%%%%%%%%%%%%%%%%%%%%%%%%%%%%%%%%%%%%%%%%%%%
%%%%%%%%%%%%%%%%%%%%%%%%%%%%%%%%%%%%%%%%%%%%%%%%%%%%%%%%
%%%%%%%%%%%%%%%%%%%%%%%%%%%%%%%%%%%%%%%%%%%%%%%%%%%%%%%%

%%%%%%%%%%%%%%%%%%%%%%%%%%%%%%%%%%%%%%%%%%%%%%%%%%%%%%%%
%%%%%%%%%%%%%%%%%%%%%%%%%%%%%%%%%%%%%%%%%%%%%%%%%%%%%%%%
%%%%%%%%%%%%%%%%%%%%%%%%%%%%%%%%%%%%%%%%%%%%%%%%%%%%%%%%

%%%%%%%%%%%%%%%%%%%%%%%%%%%%%%%%%%%%%%%%%%%%%%%%%%%%%%%%
%%%%%%%%%%%%%%%%%%%%%%%%%%%%%%%%%%%%%%%%%%%%%%%%%%%%%%%%
%%%%%%%%%%%%%%%%%%%%%%%%%%%%%%%%%%%%%%%%%%%%%%%%%%%%%%%%

%%%%%%%%%%%%%%%%%%%%%%%%%%%%%%%%%%%%%%%%%%%%%%%%%%%%%%%%
%%%%%%%%%%%%%%%%%%%%%%%%%%%%%%%%%%%%%%%%%%%%%%%%%%%%%%%%
%%%%%%%%%%%%%%%%%%%%%%%%%%%%%%%%%%%%%%%%%%%%%%%%%%%%%%%%



\subsection{Espérance et boîte magique}
%%%%%%%%%%%%%%%%%%%%%%%%%%%%%%%%%%%%%%%%%%%%%%%%%%%%%%%%%%%%%%%%%%%%%%
%%%%%%%%%%%%%%%%%%%%%%%%%%%%%%%%%%%%%%%%%%%%%%%%%%%%%%%%%%%%%%%%%%%%%%
%%%%%%%%%%%%%%%%%%%%%%%%%%%%%%%%%%%%%%%%%%%%%%%%%%%%%%%%%%%%%%%%%%%%%%%%%%%%%%%%%%%%%%%%%%%%%%%%%%%%%%%%%%%%%%%%%%%%%%%%%%%%%%%%%%%%%%%%%%%%
\subsubsection*{Enoncé du problème}
Soit une boîte magique 
agissant comme suit : 
\begin{itemize}
\item[--] La boîte "génère" : $3$, avec probabilité : $\frac{1}{3}$ ;
\item[--] La boîte "génère" : $-1$, avec probabilité : $\frac{2}{3}$.
\end{itemize}
Quelle est la somme des valeurs  
générées par la boîte, pour $n$ générations de valeur ?

%%%%%%%%%%%%%%%%%%%%%%%%%%%%%%%%%%%%%%%%%%%%%%%%%%%%%%%%
\subsubsection*{Explication théorique}

L'univers des événements pour une génération est : 
$$\Omega = \{-1, 3\}$$

Ici la variable aléatoire $X$ prend ses valeurs dans $X(\Omega) = \Omega$.

La loi de probabilité de $X$ est : 
$$
\mathcal{L}_X = \{
~~ -1 \rightarrow \frac{2}{3} ;
~~ 3 \rightarrow \frac{1}{3}
~~\}
$$

L'espérance de $X$ est ainsi :
$$\uE(X) = -1.\Pr(X=-1) + 3.\Pr(X=3) $$ 
$$\uE(X) = -1.\frac{2}{3} + 3.\frac{1}{3}$$
$$\uE(X) = \frac{1}{3}$$


Autrement dit, pour chaque valeur générée, 
on peut ``espérer'' avoir $\frac{1}{3}$.

En pratique, cela signifie que si on effectue $n$ 
générations de valeur,
alors :
si $n$ est "grand", 
la somme des valeurs générees sera ``proche'' de : 
$$n.\uE(X) = \frac{n}{3}$$

% \subsubsection{Prévision vs Expérience}

% Notons la prévsion : 
% $$p:=n.\uE(X)=\frac{n}{3}$$

% Après experience, on obtient une valeur de $r$.
% ($r$ pour ``réalisé'').

% Alors la variation $\Delta$ est définie par:
% $$
% \Delta := |p - r|
% $$

% On considère la moyenne empirique (i.e. l'espérance estimée) : 
% $$
% E_r := \frac{r}{n} (=\bar{x} )
% $$

% On pose $\Delta_E$ la variation des espérances : 
% $$
% \Delta_E := |E(X) - E_r|
% $$

% \blue{
% \begin{fTheorem}
% $$
% \Delta_E = \frac{\Delta}{n} 
% $$
% \end{fTheorem}
% }
% \begin{proof}
% $$\Delta_E = |E(X) - E_r|
% $$

% $$\Delta_E =  \frac{n.|E(X) - E_r|}{n}
% $$

% $$\Delta_E = \frac{|n.E(X) - n.E_r|}{n}
% $$

% Or :
% $$p = n.\uE(X)$$

% Et : 
% $$ E_r = \frac{r}{n} \iff  r = n.E_r $$

% $$\Delta_E = \frac{|p - r|}{n} = \frac{\Delta}{n}
% $$
% \end{proof}
%%%%%%%%%%%%%%%%%%%%%%%%%%%%%%%%%%%%%%%%%%%%%%%%%%%%%%%%
\subsubsection*{Exercice}
L'objectif est d'effectuer des simulations informatiques,
assurant (ou non), la fiabilité du résultat théorique.
