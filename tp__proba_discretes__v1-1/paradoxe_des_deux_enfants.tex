

%%%%%%%%%%%%%%%%%%%%%%%%%%%%%%%%%%%%%%%%%%%%%%%%%%%%%%%%
%%%%%%%%%%%%%%%%%%%%%%%%%%%%%%%%%%%%%%%%%%%%%%%%%%%%%%%%
%%%%%%%%%%%%%%%%%%%%%%%%%%%%%%%%%%%%%%%%%%%%%%%%%%%%%%%%
\subsection{Enoncé du problème}
Toto a exactement deux enfants.
Étant donné que l'un de ces deux enfants est un garçon,
et que la probabilité d'avoir un garçon est de $\frac{1}{2}$, 
quelle est la probabilité que l'autre enfant soit aussi un garçon ?

%%%%%%%%%%%%%%%%%%%%%%%%%%%%%%%%%%%%%%%%%%%%%%%%%%%%%%%%
%%%%%%%%%%%%%%%%%%%%%%%%%%%%%%%%%%%%%%%%%%%%%%%%%%%%%%%%
%%%%%%%%%%%%%%%%%%%%%%%%%%%%%%%%%%%%%%%%%%%%%%%%%%%%%%%%
\subsection{Résolution}
$$
\Pr(\text{2 garçons } | \text{ 1 garçon au moins}) =
 \frac {\Pr(\text{2 garçons } \wedge 
 			\text{ 1 garçon au moins})}
 		{\Pr(\text{1 garçon au moins})}
$$
$$
\Pr(\text{2 garçons } | \text{ 1 garçon au moins}) =
  \frac
 {\Pr(\text{2 garçons})}
 {\Pr(\text{1 garçon au moins})}
$$

$$
\Pr(\text{2 garçons } | \text{ 1 garçon au moins}) =
 \frac{1/4}{3/4} 
$$

$$
\Pr(\text{2 garçons } | \text{   1 garçon au moins}) =
 \frac{1}{3} 
$$


%%%%%%%%%%%%%%%%%%%%%%%%%%%%%%%%%%%%%%%%%%%%%%%%%%%%%%%%
%%%%%%%%%%%%%%%%%%%%%%%%%%%%%%%%%%%%%%%%%%%%%%%%%%%%%%%%
%%%%%%%%%%%%%%%%%%%%%%%%%%%%%%%%%%%%%%%%%%%%%%%%%%%%%%%%
\subsection{Exercice}
L'objectif est d'effectuer des simulations informatiques,
assurant (ou non), la fiabilité du résultat théorique.

