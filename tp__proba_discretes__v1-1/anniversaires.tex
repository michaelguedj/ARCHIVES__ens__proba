


\subsection{Problème des anniversaires}
Dans une classe de 25 élèves, 
quelle est la probabilité $p$, que deux élèves (au moins), 
fêtent leurs anniversaires le même jour ? 


\subsection{Intuition}
En vous basant sur votre intuition, estimer la probabilité en question.


\subsection{Considérations préalables}
On suppose :
\begin{itemize}
	\item  Que les années comportent $365$ jours par an 
	(années bissextiles non considérées) ; 
	\item La distribution des naissances est uniforme ; 
	i.e. quelque soit le jour $i\in \{1, ..., 365\}$, 
	quelque soit l'élève $x$, 
	la probabilité que $x$ fête son anniversaire le jour $i$
	est de $\frac{1}{165}$.
\end{itemize}


\subsection{Résolution du problème des anniversaires}
On considère une classe de $n$ élèves.
%%%
\begin{enumerate}
	\item Soit $n=2$. Donner la probabilité $\bar{p}$ que les deux élèves de la
	classe ne fêtent pas leurs anniversaires le même jour.
	
	\item Idem pour $n=3$.
	
	\item Idem pour $n=k$.
	
	\item 
	Résoudre le problème des anniversaires
	(utiliser la relation : $p=1-\bar{p}$).
	
	
	\item Considérer les cas : $n=30$ ; $n=50$ et $n=80$.
\end{enumerate}

\subsection{Côté individu}
Soit un individu : \textit{Toto}, dans une classe de $n=25$ élèves.
Quelle est la probabilité qu'un (autre) élève soit né le même jour que Toto ?

\begin{enumerate}
	\item Si $n=2$, quelle est la probabilité $\bar{p}$ pour que l'autre élève
	soit né un autre jour que Toto ?
	
	\item Idem pour $n=3$.
	
	\item Idem pour $n=k$.
	
	\item Trouver la probabilité $p$ 
	qu'un (autre) élève soit né le même jour que Toto 
	(utiliser la relation $p = 1-\bar{p}$).
	
	\item Que donne $p$ pour $n=30$ ? $n=50$ ? $n=80$ ?
	
\end{enumerate}

\subsection{Simulation}
On modélise une classe de $n$ élèves par une
liste de $n$ éléments compris entre $1$ et $365$.
(On suppose que la liste est indexée de $0$ à $n-1$).
\begin{enumerate}[(A)]
	\item 
	L'algorithme, ci-après, 
	permet de
	tester si deux personnes, dans la classe, sont nés le même jour.
\begin{Verbatim}
test_anniversaire(classe : liste de taille n)
begin
  for i = 0, ..., n-1 do
    for j = i+1, ..., n-1 do 
      if classe[i] = classe[j] then 
        return True
      end if
  end for; end for
  return False
end
\end{Verbatim}


\begin{fLemma}
Au $i$-ième tour de la première boucle,
en supposant que l'algorithme ne s'est pas arrêté,
$$
\forall x\in\{0, ..., i-1 \}, 
\forall j\in\{x+1, ..., n-1\}, 
classe[i] \not= classe[j]
$$
\end{fLemma}
%
\begin{proof}
Par l'absurde, 
$$
\exists x\in\{0, ..., i-1 \}, 
\exists j\in\{x+1, ..., n-1\}, 
classe[x] = classe[j]
$$
Donc au tour de boucle $x$, 
l'algorithme s'est arrêté, 
ce qui contredit notre hypothèse.
\end{proof}

\begin{fTheorem}
Au $i$-ième tour de la première boucle,
en supposant que l'algorithme ne s'est pas arrêté,
$$
	\forall j\in \{0, ..., i-1\},
	classe[i] \not= classe[j]
$$
\end{fTheorem}
	%
\begin{proof}[Preuve]
Par le lemme précédant, 
$$
\forall x\in\{0, ..., i-1 \}, 
\forall j\in\{x+1, ..., n-1\}, 
classe[i] \not= classe[j]
$$
On pose :
$$
E(x):= \{x+1, ..., n-1\}
$$
On a ainsi : 
$$
\forall x\in\{0, ..., i-1 \}, 
\forall j\in E(x), 
classe[i] \not= classe[j]
$$
On remarque que :
$$E(0) \supset E(1) \supset ... \supset E(i-1)$$
(En effet : 
$$E(0) = \{1, ..., n-1\}$$
$$E(1) = \{2, ..., n-1\}$$
$$ ... $$
$$E(i-1) = \{i, ..., n-1\}$$
)

De plus, 
$$
i\in E(i-1)
$$
Donc, 
$$
\forall x\in\{0, ..., i-1\}, i\in E(x)
$$
On rappelle que :
$$
\forall x\in\{0, ..., i-1 \}, 
\forall j\in E(x), 
classe[i] \not= classe[j]
$$
D'où, 
$$
\forall x\in\{0, ..., i-1 \}, 
classe[x] \not= classe[i]
$$
\end{proof}
	
	\item 
L'algorithme, ci-après, 
permet de
tester si une personne, dans la classe, est né le même jour que Toto.
\begin{Verbatim}
test_anniversaire_toto(classe : liste de taille n)
begin
  // Toto est supposé en position 0 (sans perte de généralité)
  for i = 1, ..., n-1 do
    if classe[0] = classe[i] then 
      return True
    end if
  end for
  return False
end
\end{Verbatim}

\item 
Via implémentation, évaluer ``pratiquement''
la justesse des résultats des sections précédentes.
\end{enumerate}