
\blue{
\begin{fDefinition}[Probabilité conditionnelle]
Une probabilité conditionnelle est la probabilité d'un événement 
sachant qu'un autre événement a eu lieu.
Soient deux événements $A$ et $B$ (i.e. $A,B\in 2^\Omega$), 
si $\Pr(A)\not=0$, alors la probabilité conditionnelle de $B$
conditionnée par $A$ (aussi appelée probabilité de $B$ sachant $A$) est définie par : 
$$
\Pr(B | A) = \frac{\Pr(A\cap B)}{\Pr(A)}
$$
\end{fDefinition}}

\noindent
\textbf{Exemple.}
Soit une classe de lycée. Nous considérons les événements suivants :
\begin{itemize}
\item[--] F : `'un élève est une fille'';
\item[--] A : ``un élève étudie l'allemand''.
\end{itemize}

La classe vérifie :
\begin{center}
\begin{tabular}{|l|c|r|}
  \hline
   & $F$ & $\neg F$ \\
  \hline
  $A$ & 10 & 7 \\
  $\neg A$ & 4 & 9 \\
  \hline
\end{tabular}
\end{center}

Quelle est la probabilité qu'un élève étudie l'allemand, sachant que cet élève est une fille
(i.e. $\Pr(A|F)$) ?
On a par définition :
$$
\Pr(A|F) = \frac{\Pr(A \wedge F)}{\Pr(F)}
$$
Ici :
$$
\Pr(A|F)= \frac{\text{Nombre de filles étudiant l'allemand}}
			 {\text{Nombre de filles}}
$$
Soit :
$$
\Pr(A|F) = \frac{10}{14} = \frac{5}{7}
$$

%%%%%%%%%%%%%%%%%%%%%%%%%%%%%%%%%%%%%%%%%%%%%%%%%%%%%%%%%%%%%%%%%%%%%
%%%%%%%%%%%%%%%%%%%%%%%%%%%%%%%%%%%%%%%%%%%%%%%%%%%%%%%%%%%%%%%%%%%%%
%%%%%%%%%%%%%%%%%%%%%%%%%%%%%%%%%%%%%%%%%%%%%%%%%%%%%%%%%%%%%%%%%%%%%
\blue{
\begin{fTheorem}
$\forall A,B\in 2^\Omega$, $\Pr(A)\not=0$,
$$
\Pr(A\cap B) = \Pr(B|A) . \Pr(A)
$$ 
\end{fTheorem}
}
\begin{proof}
$$
\Pr(B | A) = \frac{\Pr(A\cap B)}{\Pr(A)}
\iff
\Pr(B|A) . \Pr(A) = \Pr(A\cap B) 
$$
\end{proof}

%%%%%%%%%%%%%%%%%%%%%%%%%%%%%%%%%%%%%%%%%%%%%%%%%%%%%%%%%%%%%%%%%%%%%
%%%%%%%%%%%%%%%%%%%%%%%%%%%%%%%%%%%%%%%%%%%%%%%%%%%%%%%%%%%%%%%%%%%%%
%%%%%%%%%%%%%%%%%%%%%%%%%%%%%%%%%%%%%%%%%%%%%%%%%%%%%%%%%%%%%%%%%%%%%
\blue{
\begin{fLemma}
Soient $A_1, A_2 ..., A_k$ des événements tels que 
$\Pr(A_1\cap...\cap A_k) \not= 0$. Alors :
$\forall i\in\{1, ..., k\}, \Pr(A_1\cap...\cap A_i) \not= 0$.
\end{fLemma}}

\begin{proof}[Preuve]
On pose :
$$B_i := A_1\cap...\cap A_i$$

On a alors, $ \forall i\in \{1, ..., k\}$, 
$$B_k\subset B_i$$

D'où, $\forall i\in \{1, ..., k\}$, 
$\exists C \subset \Omega, 
$ tel que :
$$B_i = B_k \cup C ~ \text{ et } ~  B_k\cap C=\emptyset$$
%
Par hypothèse, $\Pr(B_k)\not=0$ et $\Pr(C) \in [0,1]$.
On en déduit : 
$$\Pr(B_i) = \Pr(B_k) + \Pr(C) \not= 0 $$ 
\end{proof}

%%%%%%%%%%%%%%%%%%%%%%%%%%%%%%%%%%%%%%%%%%%%%%%%%%%
%%%%%%%%%%%%%%%%%%%%%%%%%%%%%%%%%%%%%%%%%%%%%%%%%%%
%%%%%%%%%%%%%%%%%%%%%%%%%%%%%%%%%%%%%%%%%%%%%%%%%%%
\blue{
\begin{fTheorem}
Soient $A_1, A_2 ..., A_k$ des événements tels que :
\begin{enumerate}[(i)]
\item $k\geq 2$ ; 
\item $\Pr(A_1\cap...\cap A_{k-1})\not = 0$ ;
\end{enumerate}
Alors :
$$
\Pr(A_1\cap...\cap A_k) =
\Pr(A_1) . \Pr(A_2|A_1) . \Pr(A_3|A_1\cap A_2)
...
\Pr(A_k|A_1\cap A_2\cap...\cap A_{k-1})
$$
\end{fTheorem}
}

\begin{proof}[Preuve]
On démontre, par récurrence, 
la propriété $\Pi$, définie ci-après, pour 
$\iota\in\{2, ..., k\}$ :
$$
\Pi(\iota) : 
\Pr(A_1\cap...\cap A_\iota) =
\Pr(A_1) . \Pr(A_2|A_1) . \Pr(A_3|A_1\cap A_2)
...
\Pr(A_\iota|A_1\cap A_2\cap...\cap A_{\iota-1})
$$

\begin{enumerate}
\item \underline{Cas initial :}
Par hypothèse, et 
le lemme précédant, on a : $\Pr(A_1)\not=0$, d'où :
$$
\Pi(2) : 
\Pr(A_1\cap A_2) =
\Pr(A_1) . \Pr(A_2|A_1)
$$

\item \underline{Hypothèse de récurrence :}
Pour $\kappa \in \{2, ..., k-1\}$, 
$$\Pi(\kappa) : 
\Pr(A_1\cap...\cap A_\kappa) =
\Pr(A_1) . \Pr(A_2|A_1) . \Pr(A_3|A_1\cap A_2)
...
\Pr(A_\kappa|A_1\cap A_2\cap...\cap A_{\kappa-1})
$$

\item \underline{Hérédité :}
On a :
$$\Pr(A_1\cap...\cap A_\kappa \cap A_{\kappa+1})
=
\Pr(A_{\kappa+1} \cap (A_1\cap...\cap A_\kappa))
$$

Par hypothèse, et
le lemme précédant, 
$$\Pr(A_1\cap...\cap A_\kappa) \not=0$$
D'où, 
$$
\Pr(A_{\kappa+1} \cap (A_1\cap...\cap A_\kappa))
=
\Pr(A_{\kappa+1} | A_1\cap...\cap A_\kappa) .
\Pr(A_1\cap...\cap A_\kappa)
$$
Par hypothèse de récurrence, 
$$
\Pr(A_1\cap...\cap A_\kappa) =
\Pr(A_1) . \Pr(A_2|A_1) . \Pr(A_3|A_1\cap A_2)
...
\Pr(A_\kappa|A_1\cap A_2\cap...\cap A_{\kappa-1})
$$
D'où, 

$
\Pr(A_{\kappa+1} \cap (A_1\cap...\cap A_\kappa))
=
\Pr(A_{\kappa+1} | A_1\cap...\cap A_\kappa) \times
$
\\
$~~~~~~~~~~~~~
\Pr(A_1) . \Pr(A_2|A_1) . \Pr(A_3|A_1\cap A_2)
...
\Pr(A_\kappa|A_1\cap A_2\cap...\cap A_{\kappa-1})
$

D'où, l'établissement de $\Pi(\kappa+1)$.

Donc, en particulier, $\Pi(k)$ est vraie, selon les hypothèses.
\end{enumerate}
\end{proof}

