

%Dans ce qui suit, $(\Omega, E, P)$ est un espace probabilisé.


\blue{
\begin{fDefinition}[Partition]
$(B_i)_{i\in I}$ est une partition de $B$
($I$ est supposé fini dénombrable) ;
si :
\begin{enumerate}[(i)]
	\item $\forall i\in I, B_i\not=\emptyset$ ;
	\item $i\not= j \Rightarrow B_i \cap B_j = \emptyset$ ;
	\item $B = \bigcup_{i\in I} B_i$.
\end{enumerate}
\end{fDefinition}
}



%\begin{definition} [événements indépendants]
%Soient $A$ et $B$ sont 2 événements de probabilité non nulle.
%$A$ et $B$ sont indépendants si :
%$P(A|B) = P(A)$ ou $P(B|A) = P(B)$.
%\end{definition}
%
%\begin{lemma}
%Soit $A$ et $B$ deux évènements.
%Si $(B_i)_{i\in I}$ est une partition de $B$, alors
%les événements $A\cap B_i, i\in I$ sont indépendants.
%\end{lemma}
%
%\begin{proof}
%todo
%\end{proof}

\blue{
\begin{fTheorem}
Soit $A$ et $B$ deux évènements.
Si $(B_i)_{i\in I}$ est une partition de $B$, alors :
$$
	\Pr(A\cap B) = \sum_{i\in I} \Pr(A\cap B_i)
	$$
\end{fTheorem}
}

\begin{proof}[Preuve]
$$\Pr(A\cap B) = \Pr(A\cap \bigcup_{i\in I} B_i)
 ~~~~~~~~ ; \big( B = \bigcup_{i\in I} B_i \big)
	$$
$$\Pr(A\cap B) = 
	\Pr \big( \bigcup_{i\in I} A\cap B_i \big)
	$$
$$\Pr(A\cap B) = 
	\sum_{i\in I} \Pr(A\cap B_i)
	$$
\end{proof}

\blue{
\begin{fTheorem}
Soit $A$ un évènement.
Si $(B_i)_{i\in I}$ est une partition de $\Omega$, alors :
$$ \Pr(A) = \sum_{i\in I} \Pr(A\cap B_i)
$$
\end{fTheorem}
}

\begin{proof}[Preuve]
$$ \Pr(A) = \Pr(A\cap\Omega) = \sum_{i\in I} \Pr(A\cap B_i)
$$
\end{proof}

\blue{
\begin{fTheorem}[théorème des probabilités totales]
Soit $A$ et $B$ deux évènements.
Si $(B_i)_{i\in I}$ est une partition de l'évènement $B$, alors :
$$
\Pr(A|B) = \sum_{i\in I} \Pr(A|B_i) \Pr(B_i|B)
$$
\end{fTheorem}
}

\begin{proof}[Preuve]
$$ \Pr(A|B) = \frac{\Pr(A\cap B)}{\Pr(B)}
$$
$$ \Pr(A|B) = \frac{\sum_i \Pr(A\cap B_i)} {\Pr(B)}
$$	
$$ \Pr(A|B) = 
	\sum_i \frac{\Pr(A\cap B_i)} {\Pr(B)}
$$
$$ \Pr(A|B) = 
	\sum_i \frac{\Pr(A\cap B_i)} {\Pr(B)}
			\frac{\Pr(B_i)}{\Pr(B_i)}
$$
$$ \Pr(A|B) = 
	\sum_i \frac{\Pr(A\cap B_i)}{\Pr(B_i)}
			\frac{\Pr(B_i)}{\Pr(B)}
$$
$$ \Pr(A|B) = 
	\sum_i \Pr(A|B_i)
			\frac{\Pr(B_i)}{\Pr(B)}
$$
$$ \Pr(A|B) = 
	\sum_i \Pr(A|B_i)
			\frac{\Pr(B_i\cap B)}{\Pr(B)}
$$
$$ \Pr(A|B) = 
	\sum_i \Pr(A|B_i)
			\Pr(B_i|B)
$$
\end{proof}


\blue{
\begin{fLemma}
Soient $A$ un événement de probabilité non nulle.
Alors :
$$
\Pr(A|\Omega) = \Pr(A)
$$
\end{fLemma}
}

\begin{proof}[Preuve]
$\Pr(A|\Omega) = 
\frac{\Pr(A\cap\Omega)}{\Pr(\Omega)}
= \Pr(A)
$.
\end{proof}

\blue{
\begin{fTheorem}
\label{theoreme_proba_tot_2}
Soit $A$ un évènement.
Si $(B_i)_{i\in I}$ est une partition de $\Omega$, alors :
$$
\Pr(A) = \sum_{i\in I} \Pr(A|B_i) \Pr(B_i)
$$
\end{fTheorem}
}

\begin{proof}[Preuve]
$$
\Pr(A)  = \Pr(A|\Omega)  = \sum_{i\in I} \Pr(A|B_i) \Pr(B_i|\Omega)
$$
$$
\Pr(A)  = \sum_{i\in I} \Pr(A|B_i) \Pr(B_i)
$$
\end{proof}