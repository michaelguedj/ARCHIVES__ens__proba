
\documentclass[11pt]{article} % use larger type; default would be 10pt

%%% ----------------------------------------------------------
%%% Addings
\input{../../latex_additions/__technical_additions}
%%% End of the addings
%%% ----------------------------------------------------------



\title{Probabilités discrètes\\ -- version 1.3}
\date{\vspace{-5ex}}
\author{Dr M. GUEDJ}

\begin{document}
\renewcommand{\contentsname}{Table des Matières}
\maketitle

%%%%%%%%%%%%%%%%%%%%%%%%%%%%%% LICENCE CC %%%%%%%%%%%%%%%%%%%%%%%%%%%%
\newpage
\begin{center}
	\includegraphics[scale=0.5]{licence/licence_cc.png}
	
	\begin{small}
Probabilités discrètes de Dr Michaël GUEDJ est mis à disposition selon les termes de la licence Creative Commons Attribution 4.0 International.
Fondé(e) sur une œuvre à 
		\url{https://github.com/michaelguedj/ens__proba}.
	\end{small}
\end{center}
\newpage
%%%%%%%%%%%%%%%%%%%%%%%%%%%%%%%%%%%%%%%%%%%%%%%%%%%%%%%%%%%%%%%%%%%%%%

\tableofcontents

%%%
%%%

\section{Loi de Probabilité}
\input{loi_proba}

\section{Premiers résultats}
\blue{
\begin{fTheorem}
$\forall A\in 2^\Omega,$
$$
\Pr(A) = 1 - \Pr(\bar{A})
$$
\end{fTheorem}
}
%
\begin{proof}[Preuve]
Soit $A\in 2^\Omega$, 
on a :
$$A\cap \bar{A} = \emptyset$$
d'où par définition de $\Pr$ :
$$\Pr(A\cup \bar{A}) = \Pr(A) + \Pr(\bar{A})$$

On a : $A\cup \bar{A} = \Omega$ ; et par définition de $\Pr$ :
$\Pr(\Omega) = 1$ ; donc :
$$
\Pr(A\cup \bar{A}) = 1
$$

Par suite, 
$$1 = \Pr(A) + \Pr(\bar{A}) 
\iff 1-\Pr(\bar{A})  = \Pr(A) $$
\end{proof}


%%%%%%%%%%%%%%%%%%%%%%%%%%%%%%%%%%%%%%%%%%%%%%%%%%%%%%%%%%%%%%%%%%%%%
%%%%%%%%%%%%%%%%%%%%%%%%%%%%%%%%%%%%%%%%%%%%%%%%%%%%%%%%%%%%%%%%%%%%%
%%%%%%%%%%%%%%%%%%%%%%%%%%%%%%%%%%%%%%%%%%%%%%%%%%%%%%%%%%%%%%%%%%%%%
\blue{
\begin{fTheorem}
$$\Pr(\emptyset) = 0$$
\end{fTheorem}
}
\begin{proof}
$$
\Pr(\emptyset) = 1 - \Pr(\Omega) 
$$
\end{proof}

%%%%%%%%%%%%%%%%%%%%%%%%%%%%%%%%%%%%%%%%%%%%%%%%%%%%%%%%%%%%%%%%%%%%%
%%%%%%%%%%%%%%%%%%%%%%%%%%%%%%%%%%%%%%%%%%%%%%%%%%%%%%%%%%%%%%%%%%%%%
%%%%%%%%%%%%%%%%%%%%%%%%%%%%%%%%%%%%%%%%%%%%%%%%%%%%%%%%%%%%%%%%%%%%%
\blue{
\begin{fTheorem}
$\forall A, B \in 2^\Omega$, 
$$
\Pr(A\cup B) = \Pr(A) + \Pr(B) - \Pr(A\cap B)
$$
\end{fTheorem}
}
%
\begin{proof}[Preuve]
Soit $A, B \in 2^\Omega$ ;
On pose :
$$A' := A - (A\cap B)$$
$$B' := B - (A\cap B)$$

On a :
$A'\cap B' = \emptyset$
et donc, par définition de $\Pr$,
$$\underline{\Pr(A' \cup B') = \Pr(A') + P(B')}$$

On a :
$$(A\cap B) \cap (A'\cup B') = \emptyset$$
et donc, par définition de $\Pr$,
$$\Pr ((A\cap B) \cup (A'\cup B')) = 
\Pr(A\cap B) + \Pr(A'\cup B')
$$
En outre, 
$$A\cup B = (A\cap B) \cup (A'\cup B')$$
Donc, 
$$\Pr(A \cup B) = 
\Pr(A\cap B) + \Pr(A'\cup B')
$$
Soit, 
$$
\underline{
\Pr(A \cup B) = 
\Pr(A\cap B) + \Pr(A') + \Pr(B')
} \tab(\mE_1)
$$

On a :
$$ A = (A\cap B) \cup A' \text{ et } (A\cap B) \cap A' = \emptyset$$
$$ B = (A\cap B) \cup B' \text{ et } (A\cap B) \cap B' = \emptyset$$
D'où par définition de $\Pr$ :
$$ \Pr(A) = \Pr(A\cap B) + \Pr(A') \iff 
\underline{\Pr(A') = \Pr(A) - \Pr(A\cap B)} \tab(\mE_2)$$
$$ \Pr(B) = \Pr(A\cap B) + \Pr(B') \iff 
\underline{\Pr(B') = \Pr(B) - \Pr(A\cap B)} \tab(\mE_3)$$

Par $(\mE_1)$, $(\mE_2)$ et $(\mE_3)$, 
$$
\Pr(A\cup B) = 
	\Pr(A\cap B) + 
	\Pr(A) - \Pr(A\cap B) +
	\Pr(B) - \Pr(A\cap B)
$$
D'où
$$
\Pr(A\cup B) = 
\Pr(A)+
\Pr(B) - \Pr(A\cap B)
$$
\end{proof}

%%%%%%%%%%%%%%%%%%%%%%%%%%%%%%%%%%%%%%%%%%%%%%%%%%%%%%%%%%%%%%%%%%%%%
%%%%%%%%%%%%%%%%%%%%%%%%%%%%%%%%%%%%%%%%%%%%%%%%%%%%%%%%%%%%%%%%%%%%%
%%%%%%%%%%%%%%%%%%%%%%%%%%%%%%%%%%%%%%%%%%%%%%%%%%%%%%%%%%%%%%%%%%%%%
\blue{
\begin{fTheorem}
Soient deux événements $A, B \in 2^\Omega$, 
$$
A  \subset B \Rightarrow \Pr(A) \le \Pr(B)
$$
\end{fTheorem}
}
\begin{proof}[Preuve]
$ $
\begin{enumerate}
\item \underline{Cas 1 : $B=\emptyset$}
\\
D'où, $A=\emptyset$ ; et donc $\Pr(A)=\Pr(B)=0$.

\item \underline{Cas 2 : $B\not=\emptyset$}
\\
$\exists C\in 2^\Omega$, tel que :
$A\cup C =B$ et $A\cap C=\emptyset$.
\\
D'où, 
$$
\Pr(A\cup C) = \Pr(A) + \Pr(C) = \Pr(B)
$$
Par positivité de la probabilité, $\Pr(A) \leq \Pr(B)$.
\end{enumerate}

\end{proof}

%%%%%%%%%%%%%%%%%%%%%%%%%%%%%%%%%%%%%%%%%%%%%%%%%%%%%%%%%%%%%%%%%%%%%
%%%%%%%%%%%%%%%%%%%%%%%%%%%%%%%%%%%%%%%%%%%%%%%%%%%%%%%%%%%%%%%%%%%%%
%%%%%%%%%%%%%%%%%%%%%%%%%%%%%%%%%%%%%%%%%%%%%%%%%%%%%%%%%%%%%%%%%%%%%
\blue{
\begin{fTheorem}
Soient deux événements $A, B \in 2^\Omega$, 
$$
A  \subset B \Rightarrow \Pr(A\cap B) = \Pr(A)
$$
\end{fTheorem}
}
\begin{proof}[Preuve]
$$
A  \subset B \Rightarrow
	A\cap B = A
$$
\end{proof}

%%%%%%%%%%%%%%%%%%%%%%%%%%%%%%%%%%%%%%%%%%%%%%%%%%%%%%%%%%%%%%%%%%%%%
%%%%%%%%%%%%%%%%%%%%%%%%%%%%%%%%%%%%%%%%%%%%%%%%%%%%%%%%%%%%%%%%%%%%%
%%%%%%%%%%%%%%%%%%%%%%%%%%%%%%%%%%%%%%%%%%%%%%%%%%%%%%%%%%%%%%%%%%%%%
\blue{
\begin{fLemma}
Soit $A\in 2^\Omega$ tel que :
\begin{enumerate}[(i)]
	\item $A\subset B$ ;
	\item $B\not=\emptyset$ ;
\end{enumerate}
Alors, $\exists C\in 2^\Omega$, tel que :
\begin{enumerate}[(i)]
	\item $A\cup C = B$ ;
	\item $A\cap C = \emptyset$.
\end{enumerate}
\end{fLemma}
}
\begin{proof}[Preuve]
Sous les hypothèses, deux cas se présentent : 
\begin{enumerate}
\item $A=\emptyset$ ; alors $C=B$ ;
\item $A\not=\emptyset$ ; 
on pose : $C=\bar{A}\cap B$ ; 
en effet :
	\begin{enumerate}[(i)]
	\item 
	$
A\cup C = A\cup (\bar{A}\cap B) = 
	(A\cup \bar{A}) \cap (A\cup B) = \Omega \cap B = B 
	$ ;
	\item 
	$
A\cap C = A\cap (\bar{A}\cap B) = (A\cap \bar{A}) \cap B = 
	\emptyset \cap B = \emptyset
	$.
	\end{enumerate}
\end{enumerate}
\end{proof}




\section{Probabilité conditionnelle}

\blue{
\begin{fDefinition}[Probabilité conditionnelle]
Une probabilité conditionnelle est la probabilité d'un événement 
sachant qu'un autre événement a eu lieu.
Soient deux événements $A$ et $B$ (i.e. $A,B\in 2^\Omega$), 
si $\Pr(A)\not=0$, alors la probabilité conditionnelle de $B$
conditionnée par $A$ (aussi appelée probabilité de $B$ sachant $A$) est définie par : 
$$
\Pr(B | A) = \frac{\Pr(A\cap B)}{\Pr(A)}
$$
\end{fDefinition}}

\noindent
\textbf{Exemple.}
Soit une classe de lycée. Nous considérons les événements suivants :
\begin{itemize}
\item[--] F : `'un élève est une fille'';
\item[--] A : ``un élève étudie l'allemand''.
\end{itemize}

La classe vérifie :
\begin{center}
\begin{tabular}{|l|c|r|}
  \hline
   & $F$ & $\neg F$ \\
  \hline
  $A$ & 10 & 7 \\
  $\neg A$ & 4 & 9 \\
  \hline
\end{tabular}
\end{center}

Quelle est la probabilité qu'un élève étudie l'allemand, sachant que cet élève est une fille
(i.e. $\Pr(A|F)$) ?
On a par définition :
$$
\Pr(A|F) = \frac{\Pr(A \wedge F)}{\Pr(F)}
$$
Ici :
$$
\Pr(A|F)= \frac{\text{Nombre de filles étudiant l'allemand}}
			 {\text{Nombre de filles}}
$$
Soit :
$$
\Pr(A|F) = \frac{10}{14} = \frac{5}{7}
$$

%%%%%%%%%%%%%%%%%%%%%%%%%%%%%%%%%%%%%%%%%%%%%%%%%%%%%%%%%%%%%%%%%%%%%
%%%%%%%%%%%%%%%%%%%%%%%%%%%%%%%%%%%%%%%%%%%%%%%%%%%%%%%%%%%%%%%%%%%%%
%%%%%%%%%%%%%%%%%%%%%%%%%%%%%%%%%%%%%%%%%%%%%%%%%%%%%%%%%%%%%%%%%%%%%
\blue{
\begin{fTheorem}
$\forall A,B\in 2^\Omega$, $\Pr(A)\not=0$,
$$
\Pr(A\cap B) = \Pr(B|A) . \Pr(A)
$$ 
\end{fTheorem}
}
\begin{proof}
$$
\Pr(B | A) = \frac{\Pr(A\cap B)}{\Pr(A)}
\iff
\Pr(B|A) . \Pr(A) = \Pr(A\cap B) 
$$
\end{proof}

%%%%%%%%%%%%%%%%%%%%%%%%%%%%%%%%%%%%%%%%%%%%%%%%%%%%%%%%%%%%%%%%%%%%%
%%%%%%%%%%%%%%%%%%%%%%%%%%%%%%%%%%%%%%%%%%%%%%%%%%%%%%%%%%%%%%%%%%%%%
%%%%%%%%%%%%%%%%%%%%%%%%%%%%%%%%%%%%%%%%%%%%%%%%%%%%%%%%%%%%%%%%%%%%%
\blue{
\begin{fLemma}
Soient $A_1, A_2 ..., A_k$ des événements tels que 
$\Pr(A_1\cap...\cap A_k) \not= 0$. Alors :
$\forall i\in\{1, ..., k\}, \Pr(A_1\cap...\cap A_i) \not= 0$.
\end{fLemma}}

\begin{proof}[Preuve]
On pose :
$$B_i := A_1\cap...\cap A_i$$

On a alors, $ \forall i\in \{1, ..., k\}$, 
$$B_k\subset B_i$$

D'où, $\forall i\in \{1, ..., k\}$, 
$\exists C \subset \Omega, 
$ tel que :
$$B_i = B_k \cup C ~ \text{ et } ~  B_k\cap C=\emptyset$$
%
Par hypothèse, $\Pr(B_k)\not=0$ et $\Pr(C) \in [0,1]$.
On en déduit : 
$$\Pr(B_i) = \Pr(B_k) + \Pr(C) \not= 0 $$ 
\end{proof}

%%%%%%%%%%%%%%%%%%%%%%%%%%%%%%%%%%%%%%%%%%%%%%%%%%%
%%%%%%%%%%%%%%%%%%%%%%%%%%%%%%%%%%%%%%%%%%%%%%%%%%%
%%%%%%%%%%%%%%%%%%%%%%%%%%%%%%%%%%%%%%%%%%%%%%%%%%%
\blue{
\begin{fTheorem}
Soient $A_1, A_2 ..., A_k$ des événements tels que :
\begin{enumerate}[(i)]
\item $k\geq 2$ ; 
\item $\Pr(A_1\cap...\cap A_{k-1})\not = 0$ ;
\end{enumerate}
Alors :
$$
\Pr(A_1\cap...\cap A_k) =
\Pr(A_1) . \Pr(A_2|A_1) . \Pr(A_3|A_1\cap A_2)
...
\Pr(A_k|A_1\cap A_2\cap...\cap A_{k-1})
$$
\end{fTheorem}
}

\begin{proof}[Preuve]
On démontre, par récurrence, 
la propriété $\Pi$, définie ci-après, pour 
$\iota\in\{2, ..., k\}$ :
$$
\Pi(\iota) : 
\Pr(A_1\cap...\cap A_\iota) =
\Pr(A_1) . \Pr(A_2|A_1) . \Pr(A_3|A_1\cap A_2)
...
\Pr(A_\iota|A_1\cap A_2\cap...\cap A_{\iota-1})
$$

\begin{enumerate}
\item \underline{Cas initial :}
Par hypothèse, et 
le lemme précédant, on a : $\Pr(A_1)\not=0$, d'où :
$$
\Pi(2) : 
\Pr(A_1\cap A_2) =
\Pr(A_1) . \Pr(A_2|A_1)
$$

\item \underline{Hypothèse de récurrence :}
Pour $\kappa \in \{2, ..., k-1\}$, 
$$\Pi(\kappa) : 
\Pr(A_1\cap...\cap A_\kappa) =
\Pr(A_1) . \Pr(A_2|A_1) . \Pr(A_3|A_1\cap A_2)
...
\Pr(A_\kappa|A_1\cap A_2\cap...\cap A_{\kappa-1})
$$

\item \underline{Hérédité :}
On a :
$$\Pr(A_1\cap...\cap A_\kappa \cap A_{\kappa+1})
=
\Pr(A_{\kappa+1} \cap (A_1\cap...\cap A_\kappa))
$$

Par hypothèse, et
le lemme précédant, 
$$\Pr(A_1\cap...\cap A_\kappa) \not=0$$
D'où, 
$$
\Pr(A_{\kappa+1} \cap (A_1\cap...\cap A_\kappa))
=
\Pr(A_{\kappa+1} | A_1\cap...\cap A_\kappa) .
\Pr(A_1\cap...\cap A_\kappa)
$$
Par hypothèse de récurrence, 
$$
\Pr(A_1\cap...\cap A_\kappa) =
\Pr(A_1) . \Pr(A_2|A_1) . \Pr(A_3|A_1\cap A_2)
...
\Pr(A_\kappa|A_1\cap A_2\cap...\cap A_{\kappa-1})
$$
D'où, 

$
\Pr(A_{\kappa+1} \cap (A_1\cap...\cap A_\kappa))
=
\Pr(A_{\kappa+1} | A_1\cap...\cap A_\kappa) \times
$
\\
$~~~~~~~~~~~~~
\Pr(A_1) . \Pr(A_2|A_1) . \Pr(A_3|A_1\cap A_2)
...
\Pr(A_\kappa|A_1\cap A_2\cap...\cap A_{\kappa-1})
$

D'où, l'établissement de $\Pi(\kappa+1)$.

Donc, en particulier, $\Pi(k)$ est vraie, selon les hypothèses.
\end{enumerate}
\end{proof}



\section{Théorème des probabilités totales}


%Dans ce qui suit, $(\Omega, E, P)$ est un espace probabilisé.


\blue{
\begin{fDefinition}[Partition]
$(B_i)_{i\in I}$ est une partition de $B$
($I$ est supposé fini dénombrable) ;
si :
\begin{enumerate}[(i)]
	\item $\forall i\in I, B_i\not=\emptyset$ ;
	\item $i\not= j \Rightarrow B_i \cap B_j = \emptyset$ ;
	\item $B = \bigcup_{i\in I} B_i$.
\end{enumerate}
\end{fDefinition}
}



%\begin{definition} [événements indépendants]
%Soient $A$ et $B$ sont 2 événements de probabilité non nulle.
%$A$ et $B$ sont indépendants si :
%$P(A|B) = P(A)$ ou $P(B|A) = P(B)$.
%\end{definition}
%
%\begin{lemma}
%Soit $A$ et $B$ deux évènements.
%Si $(B_i)_{i\in I}$ est une partition de $B$, alors
%les événements $A\cap B_i, i\in I$ sont indépendants.
%\end{lemma}
%
%\begin{proof}
%todo
%\end{proof}

\blue{
\begin{fTheorem}
Soit $A$ et $B$ deux évènements.
Si $(B_i)_{i\in I}$ est une partition de $B$, alors :
$$
	\Pr(A\cap B) = \sum_{i\in I} \Pr(A\cap B_i)
	$$
\end{fTheorem}
}

\begin{proof}[Preuve]
$$\Pr(A\cap B) = \Pr(A\cap \bigcup_{i\in I} B_i)
 ~~~~~~~~ ; \big( B = \bigcup_{i\in I} B_i \big)
	$$
$$\Pr(A\cap B) = 
	\Pr \big( \bigcup_{i\in I} A\cap B_i \big)
	$$
$$\Pr(A\cap B) = 
	\sum_{i\in I} \Pr(A\cap B_i)
	$$
\end{proof}

\blue{
\begin{fTheorem}
Soit $A$ un évènement.
Si $(B_i)_{i\in I}$ est une partition de $\Omega$, alors :
$$ \Pr(A) = \sum_{i\in I} \Pr(A\cap B_i)
$$
\end{fTheorem}
}

\begin{proof}[Preuve]
$$ \Pr(A) = \Pr(A\cap\Omega) = \sum_{i\in I} \Pr(A\cap B_i)
$$
\end{proof}

\blue{
\begin{fTheorem}[théorème des probabilités totales]
Soit $A$ et $B$ deux évènements.
Si $(B_i)_{i\in I}$ est une partition de l'évènement $B$, alors :
$$
\Pr(A|B) = \sum_{i\in I} \Pr(A|B_i) \Pr(B_i|B)
$$
\end{fTheorem}
}

\begin{proof}[Preuve]
$$ \Pr(A|B) = \frac{\Pr(A\cap B)}{\Pr(B)}
$$
$$ \Pr(A|B) = \frac{\sum_i \Pr(A\cap B_i)} {\Pr(B)}
$$	
$$ \Pr(A|B) = 
	\sum_i \frac{\Pr(A\cap B_i)} {\Pr(B)}
$$
$$ \Pr(A|B) = 
	\sum_i \frac{\Pr(A\cap B_i)} {\Pr(B)}
			\frac{\Pr(B_i)}{\Pr(B_i)}
$$
$$ \Pr(A|B) = 
	\sum_i \frac{\Pr(A\cap B_i)}{\Pr(B_i)}
			\frac{\Pr(B_i)}{\Pr(B)}
$$
$$ \Pr(A|B) = 
	\sum_i \Pr(A|B_i)
			\frac{\Pr(B_i)}{\Pr(B)}
$$
$$ \Pr(A|B) = 
	\sum_i \Pr(A|B_i)
			\frac{\Pr(B_i\cap B)}{\Pr(B)}
$$
$$ \Pr(A|B) = 
	\sum_i \Pr(A|B_i)
			\Pr(B_i|B)
$$
\end{proof}


\blue{
\begin{fLemma}
Soient $A$ un événement de probabilité non nulle.
Alors :
$$
\Pr(A|\Omega) = \Pr(A)
$$
\end{fLemma}
}

\begin{proof}[Preuve]
$\Pr(A|\Omega) = 
\frac{\Pr(A\cap\Omega)}{\Pr(\Omega)}
= \Pr(A)
$.
\end{proof}

\blue{
\begin{fTheorem}
\label{theoreme_proba_tot_2}
Soit $A$ un évènement.
Si $(B_i)_{i\in I}$ est une partition de $\Omega$, alors :
$$
\Pr(A) = \sum_{i\in I} \Pr(A|B_i) \Pr(B_i)
$$
\end{fTheorem}
}

\begin{proof}[Preuve]
$$
\Pr(A)  = \Pr(A|\Omega)  = \sum_{i\in I} \Pr(A|B_i) \Pr(B_i|\Omega)
$$
$$
\Pr(A)  = \sum_{i\in I} \Pr(A|B_i) \Pr(B_i)
$$
\end{proof}

%%%%%%%%%%%%%%%%%%%%%%%%%%%
\section{Application : paradoxe de Monty Hall}


\blue{
\begin{fDefinition}[Problème de Monty Hall]
$~$
\begin{itemize}
\item Soient trois portes :
 l'une cache une voiture, chacune des deux autres cachant une chèvre.
\item Le présentateur sait où se cache la voiture.
\item Le joueur, qui souhaite trouver la voiture, 
	choisit une des portes (sans que celle-ci ne soit ouverte).
\item Le présentateur ouvre une autre porte (que celle choisie par le joueur), qui révèle alors une chèvre.
\item Le présentateur demande au candidat si celui-ci souhaite modifier son choix, 
	avant que soit effectué l'ouverture des portes.
\end{itemize}
\end{fDefinition}
}

On pose : 
\begin{itemize}
\item $ G:= $ "Le joueur gagne" ; 
\item $ 1^{Vrai} := $ "le premier choix effectué par le joueur est le bon".
\end{itemize}

\blue{
\begin{fLemma}
$$
\Pr(G) = \Pr(G|1^{Vrai}).\Pr(1^{Vrai})
	+ \Pr(G|\overline{1^{Vrai}}).\Pr(\overline{1^{Vrai}})
$$
\end{fLemma}
}

\begin{proof}[Preuve]
$\{1^{Vrai}, \overline{1^{Vrai}} \}$ est une partition de l'espace des possibles.
On utilise le théorème \ref{theoreme_proba_tot_2}.
\end{proof}


Dans ce qui suit, le problème est implicitement généralisé à $n$ portes 
(le problème initial supposant $n=3$).


\blue{
\begin{fTheorem}
$$
\Pr(G) = \Pr(G|1^{Vrai}).\frac{1}{n}
	+ \Pr(G|\overline{1^{Vrai}}).\frac{n-1}{n}
$$
\end{fTheorem}
}

\begin{proof}[Preuve]
$\Pr(1^{Vrai}) = \frac{1}{n}$ ; $\Pr(\overline{1^{Vrai}}) = \frac{n-1}{n}$.
\end{proof}

\blue{
\begin{fTheorem}
$~$
\begin{itemize}
\item[--] Si le joueur conserve son premier choix, alors $\Pr(G) = \frac{1}{n}$ ;
\item[--] Si le joueur modifie son premier choix, alors $\Pr(G) = \frac{n-1}{n}$.
\end{itemize}
\end{fTheorem}
}

\begin{proof}[Preuve]
\underline{Cas 1 : le joueur conserve son premier choix}
\begin{itemize}
\item[--] $\Pr(G|1^{Vrai}) = 1 $ ;
\item[--] $\Pr(G|\overline{1^{Vrai}}) = 0$.
\end{itemize}
D'où, 
$$ \Pr(G) = 1.\frac{1}{n}
	+ 0.\frac{n-1}{n} = \frac{1}{n}$$

\underline{Cas 2 : le joueur modifie son premier choix}
\begin{itemize}
\item[--] $\Pr(G|1^{Vrai}) = 0$ ;
\item[--] $\Pr(G|\overline{1^{Vrai}}) = 1$.
\end{itemize}
D'où, 
$$ \Pr(G) = 0.\frac{1}{n}
	+ 1.\frac{n-1}{n} = \frac{n-1}{n}$$
\end{proof}

\section{Théorème de Bayes}

\blue{
\begin{fTheorem}[Théorème de Bayes]
Soient $A$ et $B$ deux événements de probabilités non nulles.
Alors :
$$
\Pr(B|A)=\frac{\Pr(A|B) \Pr(B)}{\Pr(A)}
$$
\end{fTheorem}
}

\begin{proof}[Preuve]
$$
\Pr(B|A)=\frac{\Pr(B\cap A)}{\Pr(A)}
\iff
\Pr(A\cap B) = \Pr(B|A) . \Pr(A)
$$
$$
\Pr(A|B)=\frac{\Pr(A\cap B)}{\Pr(B)}
\iff
\Pr(A\cap B) = \Pr(A|B) . \Pr(B)
$$
D'où, 
$$
\Pr(B|A).\Pr(A) = \Pr(A|B) . \Pr(B)
\iff
\Pr(B|A) = \frac{\Pr(A|B) . \Pr(B)} {\Pr(A)}
$$
\end{proof}

%%%%%%%%%%%%%%%%%%%%%%%%%%%
%\section{Application : tests diagnostiques} 
%
\subsection{Tests diagnostiques}
\underline{Problème} :
Soit un test $t$ diagnostiquant
 la véracité d'un certain prédicat $p$.
Quelle est la probabilité qu'un individu $x$ soit vrai pour le prédicat $p$,
lorsque le test diagnostique $t$ est positif pour $x$ ?

$~\\$
\noindent
Prise de décision dans un choix à deux possibilités : positif et négatif.

$~\\$
\noindent
\begin{tabular}{| l | l | l | }
     \hline
     \textbf{Cause cherchée} 	& \textbf{Test positif}	& \textbf{Test négatif} \\ \hline
     Présence		 			& Vrai Positif (VP) 			& Faux Négatif (FN) \\ \hline
     Absence 					& Faux Positif (FP) 			& Vrai Négatif (VN) \\ \hline
\end{tabular}

%$~\\$
%Soit $x$ est une variable parcourant les individus d'une certaine
%population.

\begin{itemize}
\item $t^+(x) := \big( test(x)  = Vrai  \big)$ ;
\item $t^-(x) := \big( test(x)  = Faux  \big)$ ;
\item $p(x) := presence(x)$ ; 
\item $\overline{p(x)} := \overline{presence(x)}$.
\end{itemize}

\begin{itemize}
\item[--] $ VP = VP(test) := t^+(x) | p(x) $ ;  ($t^+(x) $ sachant $p(x) $); 
\item[--] $ FN = FN(test) := t^-(x) | p(x) $ ;
\item[--] $ FP = FP(test) := t^+(x) | \overline{p(x)} $ ;
\item[--] $ VN = VN(test) := t^-(x) | \overline{p(x)} $.
\end{itemize}

Remarquons (sous cas de la formule des probabilités totales) que : 
$$ P[VP] + P[FN] = 1$$
$$ P[FP] + P[VN] = 1$$
Mais aussi :
$$ P[VP] + P[FP] = 1$$
$$ P[FN] + P[VN] = 1$$

\begin{fTheorem}
$$
P[p(x) | t^+(x)] =
\frac{P[VP] .P[p(x)]}
{P[VP].P[p(x)] + P[FP].P[\overline{p(x)}]}
$$
\end{fTheorem}

\begin{proof}[Preuve]
$$
P[p(x) | t^+(x)] =
\frac{P[ t^+(x) | p(x) ] .P[p(x)]}
{P[t^+(x)]}
=
\frac{P[VP] .P[p(x)]}
{P[t^+(x)]}
$$

$$
P[t^+(x)] =
P[t^+(x) | p(x)].P[p(x)] + P[t^+(x) | \overline{p(x)}].P[\overline{p(x)}]
$$
$$
P[t^+(x)] =
P[VP].P[p(x)] + P[FP].P[\overline{p(x)}]
$$

$$
P[p(x) | t^+(x)] =
\frac{P[VP] .P[p(x)]}
{P[VP].P[p(x)] + P[FP].P[\overline{p(x)}]}
$$
\end{proof}

\subsection{Exemple : paradoxe du test diagnostique}
Soit un test $t$ diagnostiquant
 la véracité d'un certain prédicat $p$.
On suppose (hypothèse \textit{a priori}), que le prédicat $p$ est satisfait 
(uniformément)
pour un $\frac{1}{1000}$ des individus :
	$$P(p(X)) = \frac{1}{1000}$$
En outre, $t$ est
fiable à $99\%$ ; cela signifie que : 
\begin{enumerate}
\item 
si le test est positif pour $x$,
alors $x$
satisfait le prédicat $p$ à $\frac{99}{100}$ : 
	$$P(VP) = \frac{99}{100}$$
\item 
si le test est négatif pour $x$,
alors $x$
ne satisfait pas le prédicat $p$ à $\frac{99}{100}$ : 
	$$P(VN) = \frac{99}{100}$$
\end{enumerate}


Quelle est la probabilité qu'un  individu $x$
vérifie effectivement le prédicat $p$,
sachant que le test diagnostique $t$ est positif pour $x$ ?

$~\\$
\noindent
\underline{Réponse.}

$$
P[p(X) | t^+(X)] =
\frac{P[VP] .P[p(X)]}
{P[VP].P[p(X)] + P[FP].P[\overline{p(X)}]}
$$

$$ P[VP] + P[FP] = 1 \iff P[FP] = 1 - P[VP] $$

$$P[FP] = 1 - P[VP] = 1- \frac{99}{100} =  \frac{1}{100}$$

$$ P[p(X)] + P[\overline{p(X)}] = 1 
	\iff  P[\overline{p(X)}] = 1 - P[p(X)] = 1 - \frac{1}{1000} = \frac{999}{1000}$$

$$
P[p(X) | t^+(X)] =
\frac{ \frac{99}{100} . \frac{1}{1000}}
{\frac{99}{100} . \frac{1}{1000} + \frac{1}{100}.\frac{999}{1000}}
$$
$$
P[p(X) | t^+(X)] \simeq 9\%
$$

\section{Variable aléatoire}

\blue{
\begin{definition}[Critère]
Un critère sur $\Omega$ est une application 
$X : \Omega \rightarrow \mathbb{R}$.
\end{definition}
}

On pose :
$$X(\Omega) := \{X(w_i) : w_i\in\Omega \}$$

$X(\Omega)$ exprime : "l'ensemble des valeurs possibles du critères $X$" sur $\Omega$.

\blue{
\begin{definition}[Variable aléatoire]
Une variable aléatoire (selon le critère $X$)
est une variable parcourant  $X(\Omega)$. 
\end{definition}
}

Autrement dit, $X(\Omega)$ est l'ensemble des valeurs 
que peut prendre une variable aléatoire (selon le critère $X$).

\blue{
\begin{notation}
Un abus courant est de confondre la variable aléatoire
avec son critère : 
 $X$ est un critère sur $\Omega$, 
 tout en étant une 
variable aléatoire selon ce critère.
Nous suivons cet abus, comme il est de coutume.
Par suite, 
\begin{enumerate} 
\item $X\in X(\Omega)$ ;
\item Pour $x\in\mathbb{R}$, 
	"$X = x$" signifie : "$x$ est réalisée", i.e. :
	$$\exists w\in\Omega,X(w)=x$$ 
\item Pour $x\in\mathbb{R}$, "$\Pr(X=x)$" signifie :
	"la probabilité que $x$ soit réalisée" ; autrement dit : 
	$$
		\Pr(X=x) :=  \frac{ \#\{w\in\Omega : X(w)=x\} } {\#\Omega}
	$$
\end{enumerate} 
\end{notation}
}

Il suit que : $\forall x\in\mathbb{R}-X(\Omega)$, 
$$\Pr(X=x)=0$$


\red{
\begin{fExample}[Fil rouge]
On lance deux pièces.
$$
\Omega = \{FF, PF, FP, PP\}
$$
La variable aléatoire $X$ quantifie le nombre de pile :
$$
X(\Omega) = \{0, 1, 2 \}
$$
On a : 
$$
\Pr(X=0) = \frac{ \#\{FF\} } {4} = \frac{1}{4}
$$
$$
\Pr(X=1) = \frac{ \#\{PF, FP\} } {4} = \frac{1}{2}
$$
$$
\Pr(X=2) = \frac{ \#\{PP\} } {4} = \frac{1}{4}
$$
\end{fExample}
}

\blue{
\begin{fDefinition}[Loi de probabilité d'une variable aléatoire]
La loi de probabilité d'une variable aléatoire $X$ 
est la fonction :
$$\mathcal{L}_X : X(\Omega) \rightarrow [0,1] $$
$$x \mapsto \Pr(X=x)$$
\end{fDefinition}
}

\red{
\begin{fExample}[Fil rouge]
$$
\mathcal{L}_X = \{
~~ 0 \rightarrow \frac{1}{4};
~~ 1 \rightarrow \frac{1}{2};
~~ 2 \rightarrow \frac{1}{4}
~~ 
\}
$$
\end{fExample}
}

\blue{
\begin{fDefinition}[Espérance]
$$ \uE(X) := \sum_{i} x_i . \Pr(X=x_i)
$$
\end{fDefinition}
}

\red{
\begin{fExample}[Fil rouge]
$$\uE(X) = 0.\Pr(X=0) + 1.\Pr(X=1)  + 2.\Pr(X=2)$$ 
$$\uE(X) = 0.\frac{1}{4} + 1.\frac{1}{2} + 2.\frac{1}{4} = \frac{1}{2} + \frac{1}{2}$$
$$\uE(X) = 1 $$
Autrement dit, pour chaque lancé de 2 pièces, 
on peut ``espérer'' avoir 1 pile.
%
En pratique, cela signifie que si on effectue $n$ lancés de 2 pièces,
alors :
si $n$ est ``grand'', le nombre de piles est ``proche'' de $n$.
\end{fExample}
}

% \blue{
% \begin{fDefinition}[Variance]
% $$\uV(X) := \uE\big( (X - \uE(X))^2 \big) $$
% \end{fDefinition}
% }


% la variance d'une variable aléatoire caractérise sa capacité 
% à prendre des valeurs plus ou moins éloignées de son espérance
%La variance caractérise la dispersion d'une distribution.

% De même que l'écart type, 
% qui a l'avantage pratique
% de respecter l'ordre de grandeur.

% \blue{
% \begin{fDefinition}[Ecart-type]
% $$\sigma := \sqrt{\uV(X)}$$
% \end{fDefinition} 
% }

% \begin{fRemark}
% \begin{itemize}
% \item La variance est une mesure servant à caractériser la dispersion d'une distribution (ou d'un échantillon).
% \item De même pour l'écart type que l'on préfère dans la pratique ; 
	  % car l'écart type peut être comparé à l'ordre de grandeur des valeurs, 
	  % ce qui n'est pas le cas de la variance.
% \end{itemize}
% \end{fRemark}

% \blue{
% \begin{fTheorem}
% $$\uV(X) = \sum_{i} (x_i - \uE(X))^2  \times \Pr(X=x_i) $$
% \end{fTheorem}
% }

% \begin{proof}
% Admis. (Voir théorème de tansfert).
% \end{proof}


%\subsection{Cas des populations}
%
%Pour simplifier les preuves, 
%on réduit ici un individu à la valeur le caracterisant
%(sous entendu la valeur nous intéressant).
%
%\begin{definition}[Population]
%Une population est un triplet $(\frak{P}, \frak{V}, val)$
%tel que : 
%\begin{itemize}
%\item $\frak{P}$ est un ensemble d'individus :
%$$\frak{P}=\{ \frak{p}_1, ..., \frak{p}_n \}$$
%\item $\frak{V}\subset \mathbb{R}$ est un ensemble de valeurs ; 
%\item $val : \frak{P} \rightarrow \frak{V}$. 
%\end{itemize}
%\end{definition}
%
%Soit la fonction $val : \frak{P} \rightarrow X(\Omega)$, 
%associant à chaque individu une valeur.
%
%\begin{definition}[Population]
%Une population $\frak{P}$ est un multi-ensemble (de valeurs) :
%$$\frak{P}=< \frak{p}_1, ..., \frak{p}_n >$$. 
%\end{definition}
%
%\begin{definition}[Moyenne d'une population ou moyenne réelle]
%$$\mu = \frac{1}{n}\sum_{i=1}^n \frak{p}_i$$
%\end{definition}
%
%\begin{definition}[Moyenne d'une population ou moyenne réelle]
%$$\mu = \frac{1}{n}\sum_{i=1}^n val(\frak{p}_i)$$
%\end{definition}
%
%\begin{definition}
%$X\in\{ \frak{p}_i: i=1, ..., n\} = \{x_i, i\}$
%est une variable aléatoire caracterisant la population $\frak{P}$.
%\end{definition}
%
%\begin{proposition}
%$$
%E(X) = \mu
%$$
%\end{proposition}
%
%\begin{proof}[Preuve]
%
%$$
%\tilde{x_i} := \{ \frak{p}_j : \frak{p}_j = x_i\}
%$$
%
%$$
%n_i := | \tilde{x_i} |
%$$
%
%
%
%$$
%P(X=x_i) = \frac{n_i}{n}
%$$ 
%
%$$
%E(X) = \sum_{x_i\in X(\Omega)} x_i . P(X=x_i)
%$$
%
%$$
%E(X) = \sum_{x_i\in X(\Omega)} 
%	x_i . 
%	\frac{n_i}{n}
%$$
%
%$$
%E(X) = \sum_{x_i\in X(\Omega)} 
%		\sum_{\frak{p}\in \tilde{x_i}}
%						\frak{p}.\frac{1}{n}
%$$
%
%$$
%E(X) = \frac{1}{n}
%	   \sum_{x_i\in X(\Omega)} 
%		\sum_{\frak{p}\in \tilde{x_i}}
%						\frak{p}
%$$
%
%$$
%E(X) = \frac{1}{n} \sum_{i=1}^{n} \frak{p}_i 
%$$
%
%
%$$
%n_i := | \{ \frak{p}_j : val(\frak{p}_j) = x_i\}|
%$$
%
%$$
%p_{x_i} := \{ \frak{p}_j : val(\frak{p}_j) = x_i\}
%$$
%
%$$
%P(X=x_i) = \frac{n_i}{n}
%$$ 
%
%$$
%E(X) = \sum_{x_i\in X(\Omega)} x_i . P(X=x_i)
%$$
%
%$$
%E(X) = \sum_{x_i\in X(\Omega)} 
%	x_i . 
%	\frac{n_i}{n}
%$$
%
%$$
%E(X) = \sum_{x_i\in X(\Omega)} 
%		\sum_{\frak{p}\in \frak{p}_{x_i}}
%						val(\frak{p}).\frac{1}{n}
%$$
%
%$$
%E(X) = \frac{1}{n}
%	   \sum_{x_i\in X(\Omega)} 
%		\sum_{\frak{p}\in \frak{p}_{x_i}}
%						val(\frak{p})
%$$
%
%$$
%E(X) = \frac{1}{n} \sum_{i=1}^{n} val(\frak{p}_i) 
%$$
%
%\end{proof}
%
%\begin{proposition}[Variance réelle]
%$$
%V(X) = \frac{1}{n} \sum_{i=1}^{n} (x_i - \mu)^2
%$$
%\end{proposition}
%
%\begin{proof}
%
%Considérons le multi-ensemble :
%$$\frak{P}'
%:= <(x_i-\mu)^2 : i=1, ..., n>
%$$
%
%Considérons la population 
%$\frak{P}' := \{ \frak{p}'_i : i=1, ..., n \}
%$.
%La fonction $val':\frak{P}' \rightarrow Y(\Omega)$
%associe à chaque individu une valeur.
%On pose : 
%$$
%Y(\Omega) := \{(x_i - \mu)^2 : i=1, ..., n\} 
%\text{ (scholie :  ensemble et non multi-ensemble !) }
%$$
%
%On a :
%$$
%E(Y) = \mu_{val'} = \sum_{i=1}^n val'(\frak{p}'_i)
%$$
%
%\end{proof}

%%%%%%%%%%%%%%%%%%%%%%%%%%%
%%%%%%%%%%%%%%%%%%%%%%%%%%%
%%%%%%%%%%%%%%%%%%%%%%%%%%%
%%%%%%%%%%%%%%%%%%%%%%%%%%%
%%%%%%%%%%%%%%%%%%%%%%%%%%%
%%%%%%%%%%%%%%%%%%%%%%%%%%%
%\section{Lois usuelles discrètes}
%Dans ce qui suit $|X(\Omega)|=n$.
%\subsection{Loi uniforme}
%\begin{definition}
%$X$ suit une loi uniforme de paramètre $n$ si :
%$$\forall x_i\in X(\Omega), P(X=x_i) = \frac{1}{n}$$
%(les valeurs prises par la variable aléatoire sont équiprobables).
%\end{definition}
%
%\begin{proposition}
%Si $X(\Omega) = \{1, ..., n\}$, 
%Alors
% $$E(X) = \frac{n+1}{n}$$
%\end{proposition}
%
%%\begin{proof}[Preuve]
%%$$E(X) = \sum_{i=1}^{n} x_i P(X=x_i) = \sum_{i=1}^{n} x_i .\frac{1}{n}
%%=\frac{1}{n} \sum_{i=1}^{n} x_i$$
%%Selon l'hypothèse $X(\Omega) = \{1, ..., n\}$ :
%%$$E(X) =\frac{1}{n} \sum_{i=1}^{n} i$$
%%Rappelons que :
%%$$ \sum_{i=1}^{n} i = \frac{(n+1).n}{2}$$
%%D'où, 
%%$$E(X) =\frac{1}{n} . \frac{(n+1).n}{2}$$
%%$$E(X) = \frac{n+1}{2} $$
%%\end{proof}
%
%
%\begin{tabular}{|l|l|l|l|l|}
%  \hline
%  Loi & $X(\Omega)$ & $P(X=k)$ & E(X) & V(X) \\ \hline
%  \hline
%  Uniforme & \{1, ..., n\} & 1.3 & 0 & 0 \\ \hline
%  Bernouilli & \{0, ..., n\} & 2.3 & 0 & 0 \\ \hline
%  Binomiale & \{0, ..., n\} & 2.3 & 0 & 0 \\ \hline
%  Poisson & $\mathbb{N}$ & 2.3 & 0 & 0 \\ \hline
%\end{tabular}
%
%\subsection{Loi Bernoulli}
%\subsection{Loi Binomiale}
%\subsection{Loi Poisson}



%%%%%%%%%%%%%%%%%%%%%%%%%%%
%%%%%%%%%%%%%%%%%%%%%%%%%%%
%%%%%%%%%%%%%%%%%%%%%%%%%%%
%%%%%%%%%%%%%%%%%%%%%%%%%%%
%%%%%%%%%%%%%%%%%%%%%%%%%%%
%%%%%%%%%%%%%%%%%%%%%%%%%%%
%\section{todo: stat de base }
%variance empirique
%\begin{definition}[variance, écart type]
%$~$
%\begin{itemize}
%\item Variance : 
%\item Ecart type : $\sigma (X) = \sqrt{V(X)}$
%\end{itemize}
%\end{definition}
%
%Indicateurs statistiques courrant.

%%%%%%%%%%%%%%%%%%%%%%%%%%%
%%%%%%%%%%%%%%%%%%%%%%%%%%%
%%%%%%%%%%%%%%%%%%%%%%%%%%%
%%%%%%%%%%%%%%%%%%%%%%%%%%%
%%%%%%%%%%%%%%%%%%%%%%%%%%%
%%%%%%%%%%%%%%%%%%%%%%%%%%%

\end{document}