

\blue{
\begin{fDefinition}[Problème de Monty Hall]
$~$
\begin{itemize}
\item Soient trois portes :
 l'une cache une voiture, chacune des deux autres cachant une chèvre.
\item Le présentateur sait où se cache la voiture.
\item Le joueur, qui souhaite trouver la voiture, 
	choisit une des portes (sans que celle-ci ne soit ouverte).
\item Le présentateur ouvre une autre porte (que celle choisie par le joueur), qui révèle alors une chèvre.
\item Le présentateur demande au candidat si celui-ci souhaite modifier son choix, 
	avant que soit effectué l'ouverture des portes.
\end{itemize}
\end{fDefinition}
}

On pose : 
\begin{itemize}
\item $ G:= $ "Le joueur gagne" ; 
\item $ 1^{Vrai} := $ "le premier choix effectué par le joueur est le bon".
\end{itemize}

\blue{
\begin{fLemma}
$$
\Pr(G) = \Pr(G|1^{Vrai}).\Pr(1^{Vrai})
	+ \Pr(G|\overline{1^{Vrai}}).\Pr(\overline{1^{Vrai}})
$$
\end{fLemma}
}

\begin{proof}[Preuve]
$\{1^{Vrai}, \overline{1^{Vrai}} \}$ est une partition de l'espace des possibles.
On utilise le théorème \ref{theoreme_proba_tot_2}.
\end{proof}


Dans ce qui suit, le problème est implicitement généralisé à $n$ portes 
(le problème initial supposant $n=3$).


\blue{
\begin{fTheorem}
$$
\Pr(G) = \Pr(G|1^{Vrai}).\frac{1}{n}
	+ \Pr(G|\overline{1^{Vrai}}).\frac{n-1}{n}
$$
\end{fTheorem}
}

\begin{proof}[Preuve]
$\Pr(1^{Vrai}) = \frac{1}{n}$ ; $\Pr(\overline{1^{Vrai}}) = \frac{n-1}{n}$.
\end{proof}

\blue{
\begin{fTheorem}
$~$
\begin{itemize}
\item[--] Si le joueur conserve son premier choix, alors $\Pr(G) = \frac{1}{n}$ ;
\item[--] Si le joueur modifie son premier choix, alors $\Pr(G) = \frac{n-1}{n}$.
\end{itemize}
\end{fTheorem}
}

\begin{proof}[Preuve]
\underline{Cas 1 : le joueur conserve son premier choix}
\begin{itemize}
\item[--] $\Pr(G|1^{Vrai}) = 1 $ ;
\item[--] $\Pr(G|\overline{1^{Vrai}}) = 0$.
\end{itemize}
D'où, 
$$ \Pr(G) = 1.\frac{1}{n}
	+ 0.\frac{n-1}{n} = \frac{1}{n}$$

\underline{Cas 2 : le joueur modifie son premier choix}
\begin{itemize}
\item[--] $\Pr(G|1^{Vrai}) = 0$ ;
\item[--] $\Pr(G|\overline{1^{Vrai}}) = 1$.
\end{itemize}
D'où, 
$$ \Pr(G) = 0.\frac{1}{n}
	+ 1.\frac{n-1}{n} = \frac{n-1}{n}$$
\end{proof}