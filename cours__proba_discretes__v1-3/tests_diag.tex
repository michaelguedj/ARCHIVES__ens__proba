
\subsection{Tests diagnostiques}
\underline{Problème} :
Soit un test $t$ diagnostiquant
 la véracité d'un certain prédicat $p$.
Quelle est la probabilité qu'un individu $x$ soit vrai pour le prédicat $p$,
lorsque le test diagnostique $t$ est positif pour $x$ ?

$~\\$
\noindent
Prise de décision dans un choix à deux possibilités : positif et négatif.

$~\\$
\noindent
\begin{tabular}{| l | l | l | }
     \hline
     \textbf{Cause cherchée} 	& \textbf{Test positif}	& \textbf{Test négatif} \\ \hline
     Présence		 			& Vrai Positif (VP) 			& Faux Négatif (FN) \\ \hline
     Absence 					& Faux Positif (FP) 			& Vrai Négatif (VN) \\ \hline
\end{tabular}

%$~\\$
%Soit $x$ est une variable parcourant les individus d'une certaine
%population.

\begin{itemize}
\item $t^+(x) := \big( test(x)  = Vrai  \big)$ ;
\item $t^-(x) := \big( test(x)  = Faux  \big)$ ;
\item $p(x) := presence(x)$ ; 
\item $\overline{p(x)} := \overline{presence(x)}$.
\end{itemize}

\begin{itemize}
\item[--] $ VP = VP(test) := t^+(x) | p(x) $ ;  ($t^+(x) $ sachant $p(x) $); 
\item[--] $ FN = FN(test) := t^-(x) | p(x) $ ;
\item[--] $ FP = FP(test) := t^+(x) | \overline{p(x)} $ ;
\item[--] $ VN = VN(test) := t^-(x) | \overline{p(x)} $.
\end{itemize}

Remarquons (sous cas de la formule des probabilités totales) que : 
$$ P[VP] + P[FN] = 1$$
$$ P[FP] + P[VN] = 1$$
Mais aussi :
$$ P[VP] + P[FP] = 1$$
$$ P[FN] + P[VN] = 1$$

\begin{fTheorem}
$$
P[p(x) | t^+(x)] =
\frac{P[VP] .P[p(x)]}
{P[VP].P[p(x)] + P[FP].P[\overline{p(x)}]}
$$
\end{fTheorem}

\begin{proof}[Preuve]
$$
P[p(x) | t^+(x)] =
\frac{P[ t^+(x) | p(x) ] .P[p(x)]}
{P[t^+(x)]}
=
\frac{P[VP] .P[p(x)]}
{P[t^+(x)]}
$$

$$
P[t^+(x)] =
P[t^+(x) | p(x)].P[p(x)] + P[t^+(x) | \overline{p(x)}].P[\overline{p(x)}]
$$
$$
P[t^+(x)] =
P[VP].P[p(x)] + P[FP].P[\overline{p(x)}]
$$

$$
P[p(x) | t^+(x)] =
\frac{P[VP] .P[p(x)]}
{P[VP].P[p(x)] + P[FP].P[\overline{p(x)}]}
$$
\end{proof}

\subsection{Exemple : paradoxe du test diagnostique}
Soit un test $t$ diagnostiquant
 la véracité d'un certain prédicat $p$.
On suppose (hypothèse \textit{a priori}), que le prédicat $p$ est satisfait 
(uniformément)
pour un $\frac{1}{1000}$ des individus :
	$$P(p(X)) = \frac{1}{1000}$$
En outre, $t$ est
fiable à $99\%$ ; cela signifie que : 
\begin{enumerate}
\item 
si le test est positif pour $x$,
alors $x$
satisfait le prédicat $p$ à $\frac{99}{100}$ : 
	$$P(VP) = \frac{99}{100}$$
\item 
si le test est négatif pour $x$,
alors $x$
ne satisfait pas le prédicat $p$ à $\frac{99}{100}$ : 
	$$P(VN) = \frac{99}{100}$$
\end{enumerate}


Quelle est la probabilité qu'un  individu $x$
vérifie effectivement le prédicat $p$,
sachant que le test diagnostique $t$ est positif pour $x$ ?

$~\\$
\noindent
\underline{Réponse.}

$$
P[p(X) | t^+(X)] =
\frac{P[VP] .P[p(X)]}
{P[VP].P[p(X)] + P[FP].P[\overline{p(X)}]}
$$

$$ P[VP] + P[FP] = 1 \iff P[FP] = 1 - P[VP] $$

$$P[FP] = 1 - P[VP] = 1- \frac{99}{100} =  \frac{1}{100}$$

$$ P[p(X)] + P[\overline{p(X)}] = 1 
	\iff  P[\overline{p(X)}] = 1 - P[p(X)] = 1 - \frac{1}{1000} = \frac{999}{1000}$$

$$
P[p(X) | t^+(X)] =
\frac{ \frac{99}{100} . \frac{1}{1000}}
{\frac{99}{100} . \frac{1}{1000} + \frac{1}{100}.\frac{999}{1000}}
$$
$$
P[p(X) | t^+(X)] \simeq 9\%
$$