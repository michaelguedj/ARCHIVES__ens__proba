
\documentclass[11pt]{article} % use larger type; default would be 10pt

%%% ----------------------------------------------------------
%%% Addings
\input{../../latex_additions/__technical_additions}
%%% End of the addings
%%% ----------------------------------------------------------



\title{Probabilités discrètes -- Résumé\\ -- version 0.1}
\date{\vspace{-5ex}}
\author{Dr M. GUEDJ}

\begin{document}
\maketitle

%%%%%%%%%%%%%%%%%%%%%%%%%%%%%% LICENCE CC %%%%%%%%%%%%%%%%%%%%%%%%%%%%
\begin{center}
	\includegraphics[scale=0.5]{licence/licence_cc.png}
	
	\begin{small}
``Probabilités discrètes -- Résumé'' de Dr Michaël GUEDJ est mis à disposition selon les termes de la licence Creative Commons Attribution 4.0 International.
Fondé(e) sur une œuvre à 
		\url{https://github.com/michaelguedj/ens__proba}.
	\end{small}
\end{center}
%%%%%%%%%%%%%%%%%%%%%%%%%%%%%%%%%%%%%%%%%%%%%%%%%%%%%%%%%%%%%%%%%%%%%%

%%%%%%%%%%%%%%%%%%%%%%%%%%%%%%%%
%%%%%%%%%%%%%%%%%%%%%%%%%%%%%%%%
%%%%%%%%%%%%%%%%%%%%%%%%%%%%%%%%
\section{Loi de probabilité}
\blue{
\begin{fDefinition}[Loi de probabilité]
Une loi de probabilité (ou probabilité) $\Pr$,
sur $2^\Omega$,
est une application vérifiant : 
\begin{enumerate}
	\item Masse unitaire : $\Pr(\Omega)=1$ ;
	\item Positivité : $\Pr: 2^\Omega \rightarrow [0, 1]$ ;
	\item Additivité : $\forall A,B\in2^\Omega$, 
			$$A\cap B=\emptyset \Rightarrow 
			\Pr(A\cup B) = \Pr(A) + \Pr(B)$$
\end{enumerate}
\end{fDefinition}
}
Dans ce qui suit $\Pr$ est une probabilité sur $\Omega$.

\blue{
\begin{fTheorem}
$\forall A\in 2^\Omega,$
$$
\Pr(A) = 1 - \Pr(\bar{A})
$$
\end{fTheorem}
}

\blue{
\begin{fTheorem}
$\forall A, B \in 2^\Omega$, 
$$
\Pr(A\cup B) = \Pr(A) + \Pr(B) - \Pr(A\cap B)
$$
\end{fTheorem}
}

%%%%%%%%%%%%%%%%%%%%%%%%%%%%%%%%
%%%%%%%%%%%%%%%%%%%%%%%%%%%%%%%%
%%%%%%%%%%%%%%%%%%%%%%%%%%%%%%%%
\section{Probabilité conditionnelle}
\blue{
\begin{fDefinition}[Probabilité conditionnelle]
Une probabilité conditionnelle est la probabilité d'un événement 
sachant qu'un autre événement a eu lieu.
Soient deux événements $A$ et $B$ (i.e. $A,B\in 2^\Omega$), 
si $\Pr(A)\not=0$, alors la probabilité conditionnelle de $B$
conditionnée par $A$ (aussi appelée probabilité de $B$ sachant $A$) est définie par : 
$$
\Pr(B | A) = \frac{\Pr(A\cap B)}{\Pr(A)}
$$
\end{fDefinition}}

\noindent
\textbf{Exemple.}
Soit une classe de lycée. Nous considérons les événements suivants :
\begin{itemize}
\item[--] F : `'un élève est une fille'';
\item[--] A : ``un élève étudie l'allemand''.
\end{itemize}

La classe vérifie :
\begin{center}
\begin{tabular}{|l|c|r|}
  \hline
   & $F$ & $\neg F$ \\
  \hline
  $A$ & 10 & 7 \\
  $\neg A$ & 4 & 9 \\
  \hline
\end{tabular}
\end{center}

Quelle est la probabilité qu'un élève étudie l'allemand, sachant que cet élève est une fille
(i.e. $\Pr(A|F)$) ?
On a par définition :
$$
\Pr(A|F) = \frac{\Pr(A \wedge F)}{\Pr(F)}
$$
Ici :
$$
\Pr(A|F)= \frac{\text{Nombre de filles étudiant l'allemand}}
			 {\text{Nombre de filles}}
$$
Soit :
$$
\Pr(A|F) = \frac{10}{14} = \frac{5}{7}
$$

\blue{
\begin{fTheorem}
$\forall A,B\in 2^\Omega$, $\Pr(A)\not=0$,
$$
\Pr(A\cap B) = \Pr(B|A) . \Pr(A)
$$ 
\end{fTheorem}
}

%%%%%%%%%%%%%%%%%%%%%%%%%%%%%%%%
%%%%%%%%%%%%%%%%%%%%%%%%%%%%%%%%
%%%%%%%%%%%%%%%%%%%%%%%%%%%%%%%%
\section{Théorème des probabilités totales}
\blue{
\begin{fTheorem}[Théorème des probabilités totales]
Soit $A$ un évènement.
Si $(B_i)_{i\in I}$ est une partition de $\Omega$, alors :
$$
\Pr(A) = \sum_{i\in I} \Pr(A|B_i) \Pr(B_i)
$$
\end{fTheorem}
}

%%%%%%%%%%%%%%%%%%%%%%%%%%%%%%%%
%%%%%%%%%%%%%%%%%%%%%%%%%%%%%%%%
%%%%%%%%%%%%%%%%%%%%%%%%%%%%%%%%
\section{Théorème de Bayes}
\blue{
\begin{fTheorem}[Théorème de Bayes]
Soient $A$ et $B$ deux événements de probabilités non nulles.
Alors :
$$
\Pr(B|A)=\frac{\Pr(A|B) \Pr(B)}{\Pr(A)}
$$
\end{fTheorem}
}

%%%%%%%%%%%%%%%%%%%%%%%%%%%%%%%%
%%%%%%%%%%%%%%%%%%%%%%%%%%%%%%%%
%%%%%%%%%%%%%%%%%%%%%%%%%%%%%%%%
\section{Variable aléatoire -- Espérance}
\red{
\begin{fExample}[Fil rouge]
On lance deux pièces.
$$
\Omega = \{FF, PF, FP, PP\}
$$
La variable aléatoire $X$ quantifie le nombre de pile :
$$
X(\Omega) = \{0, 1, 2 \}
$$
On a : 
$$
\Pr(X=0) = \frac{ \#\{FF\} } {4} = \frac{1}{4}
$$
$$
\Pr(X=1) = \frac{ \#\{PF, FP\} } {4} = \frac{1}{2}
$$
$$
\Pr(X=2) = \frac{ \#\{PP\} } {4} = \frac{1}{4}
$$
\end{fExample}
}

\red{
\begin{fExample}[Fil rouge]
$$
\mathcal{L}_X = \{
~~ 0 \rightarrow \frac{1}{4};
~~ 1 \rightarrow \frac{1}{2};
~~ 2 \rightarrow \frac{1}{4}
~~ 
\}
$$
\end{fExample}
}

\blue{
\begin{fDefinition}[Espérance]
$$ \uE(X) := \sum_{i} x_i . \Pr(X=x_i)
$$
\end{fDefinition}
}

\red{
\begin{fExample}[Fil rouge]
$$\uE(X) = 0.\Pr(X=0) + 1.\Pr(X=1)  + 2.\Pr(X=2)$$ 
$$\uE(X) = 0.\frac{1}{4} + 1.\frac{1}{2} + 2.\frac{1}{4} = \frac{1}{2} + \frac{1}{2}$$
$$\uE(X) = 1 $$
Autrement dit, pour chaque lancé de 2 pièces, 
on peut ``espérer'' avoir 1 pile.
%
En pratique, cela signifie que si on effectue $n$ lancés de 2 pièces,
alors :
si $n$ est ``grand'', le nombre de piles est ``proche'' de $n$.
\end{fExample}
}
\end{document}